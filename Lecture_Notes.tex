\documentclass[a4paper,11pt]{article}
%\usepackage[utf8]{inputenc}

\usepackage{amsmath, amsfonts, amsthm, amssymb}
\usepackage{tikz}
\usepackage{braket}
\usepackage{cancel}
\usepackage{xcolor}
\usepackage{relsize}

\renewcommand{\d}{{\mathrm{d}}}
\newcommand{\D}{{\mathcal{D}}}
\newcommand{\e}{{\mathrm{e}}}
\newcommand{\dr}[2]{\frac{\partial {#1}}{\partial{#2}}}
\newcommand{\ppi}{{\mathlarger{\mathlarger{\mathlarger{\pi}}}}}



\usepackage{geometry}
\geometry{a4paper, left=25mm, right=25mm, top=30mm, bottom=25mm}

\renewcommand{\baselinestretch}{1.2}

\begin{document}
\title{Notes de cours de\\Physique-Mathématique et Géométrie Différentielle}
\author{cours de Frédérique Hélein - notes de Sacha Amiel}
\maketitle
\tableofcontents
\newpage
\noindent\underline{Objectif}: Atteindre les théories BRST\footnote{BRST: Carlo Becchi, Alain Rouet, Raymond Stora \& Igor Tyutin} et BV\footnote{Igor Batalin \& Grigori Vilkovisky}, théories physiques dévellopées pour quantifier les théories de jauge, tout particulièrement les Yang-Mills mais aussi d'autres.

\section{Calcul des variations}
\subsection{Rappels de base de physique des particules classiques (en formalisme Lagrangien)}
\quad On considère une particule (classique) dans une variété $\mathcal{M}$ de dimension $m$; $I=]t_0,t_1[$ un intervalle réel ouvert (le temps, $t_0, t_1 \in \mathbb{R}\cup \{\pm \infty\}$); et on appel:
\begin{align*}
	``\mathrm{Lagrangien}":
		&&L: &&I\times\mathcal{M} \quad&\to\quad \mathbb{R}\\
		&&&& (t,x,v) \quad&\mapsto\quad L(t,x,v)\notag\\
	``\mathrm{Action}":
		&&\mathcal{A}: &&\mathcal{C}^1(I,\mathcal{M}) \quad &\to \quad	\mathbb{R}\\
		&&&&\gamma \quad&\mapsto\quad\mathcal{A}[\gamma]:=\int_I L\big(t,\gamma(t),\dot \gamma(t)\big)\notag
\end{align*}
où $L$ est au moins $\mathcal{C}^1$ en $x$ et $\mathcal{C}^2$ en $v$, et où
\begin{equation*}\begin{split}
\delta\mathcal{A}_\gamma[\delta\gamma]
&=\int_I \dr{L}{x^i}(t,\gamma,\dot \gamma)\delta\gamma^i + \dr{L}{v^i}(t,\gamma, \dot \gamma)\frac{\d \delta \gamma^i}{\d t}\\
&=\int_I \frac{\d}{\d t}\left(\dr{L}{v^i}(t,\gamma,\dot \gamma)\delta\gamma^i \right)+ \left(\dr{L}{x^i}(t,\gamma, \dot \gamma) - \frac{\d}{\d t}\dr{L}{v^i}\right)\delta \gamma^i
\end{split}\end{equation*}

\underline{Principe de Maupertuis} (généralisé): On obtient les trajectoires d'une physique classique régie par $L$ en se restreignant à l'ensemble des chemins $\gamma$ tels que $\forall \delta \gamma \quad \delta\mathcal{A}_\gamma[\delta_\gamma]=0$. i.e. ce sont les chemins qui extremisent localement l'action (hors cas physique, on parlera donc simplement de ``points critiques").\\ \\
D'où on dérive le \underline{principe d'Hamiltion}: 
$\forall \delta\gamma \; \mathrm{t}.\mathrm{q}.\; \delta\gamma(t_0)=\delta\gamma(t_1)=0$
\begin{equation*}
\;_{(\mathrm{Maup})} \; \delta\mathcal{A}_\gamma[\delta\gamma] = 0 \quad \quad \iff \quad \quad \boxed{\frac{\d}{\d t}\left(\dr{L}{v_i}(t,\gamma,\dot\gamma)\right) = \dr{L}{x^i}(t,\gamma,\dot\gamma)}_{\quad(\mathrm{E}.\scalebox{0.75}[1.0]{-}\mathrm{L}.)}
\end{equation*}
où l'équation à droite est appelée ``equations d'\underline{Euler-Lagrange}" (E.-L.) (pour une physique de particules). (Existe aussi en version théorie de champs, cf plus tard).

\subsection{1$^\mathrm{er}$ théorème de Noether, symétries et conservation (cas des particules)}
Première difficulté: qu'est-ce qu'une symétrie? Il s'agit, grossièrement d'une action d'un groupe de Lie. (Enfin, d'une algèbre de Lie plutôt...)\\
Version simple:

\color{red}METTRE LE DESSIN\color{black}

$$X = X^0(t,x)\frac{\partial}{\partial t} + X^i(t,x) \frac{\partial}{\partial x^i} \quad \quad \quad T:=X^0$$
On note $\Delta_X\subset \mathbb{R}\times(I\times\mathcal{M})$ maximal sur lequel le flot est défini.
\begin{equation*}
	\Phi_X : \left\{\begin{matrix}
	\Delta_X & \to & I\times\mathcal{M} &\\
	(\epsilon,t,x) & \mapsto & \Phi_X(\epsilon,t,x) &=\e^{\epsilon X}(t,x)
	\end{matrix}\right.
\end{equation*}
i.e. $\frac{\partial \Phi}{\partial \epsilon}(\epsilon, t, x) = X(\Phi_X(\epsilon, t, x))$ et $\Phi_X(0,t,x)=(t,x)$
\\
en coord loc, ça donne: $\e^{\epsilon X}(t,x)=\left(\quad t+\epsilon T(t,x),\quad x^i+\epsilon X^i(t,x)\quad \right) + o(\epsilon)$
\\
Action sur $\mathcal{C}^1(I',\mathcal{M})$ où $I'$ est un intervalle compacte de $I$:
\begin{align*}
\gamma &\mapsto \gamma_\epsilon\\
[t_0,t_1] &\mapsto[t_0(\epsilon),t_1(\epsilon)]=[t_0+\epsilon T(t_0,x_0),t_1+\epsilon T(t_1,x_1)]\quad \mathrm{modulo}\;\epsilon
\end{align*}
\begin{align*}
\forall i \in [\![1,n]\!] \quad \gamma^i_{\epsilon}(\Phi_X^0(\epsilon,t,x) &= \Phi_X^i(\epsilon,t,\gamma(t))\\
\gamma_\epsilon &= \gamma + \epsilon\delta\gamma + o(\epsilon)
\end{align*}
\begin{align*}
(\gamma^i+\epsilon\delta\gamma^i)(t+\epsilon T(t,\gamma)) = \gamma^i + \epsilon X^i(t,\gamma) + o(\epsilon)
&\iff \frac{\d \gamma^i}{\d t}T + \delta\gamma^i = X^i\\
&\iff \boxed{\delta \gamma^i = X^i(t,\gamma) - T(t,\gamma)\dot\gamma^i}
\end{align*}
$$X \; \mathrm{symetrie}\;\mathrm{de}\;L \overset{(\mathrm{def})}{\iff} \forall [t_0,t_1]\subset I \quad \int_{t_0(\epsilon)}^{t_1(\epsilon)} L(t,\gamma_\epsilon, \dot\gamma_\epsilon)\d t = \int_{t_0}^{t_1} L(t,\gamma,\dot\gamma)\d t + (\epsilon)$$
\underline{Théorème 1}: Si $X$ est une symétrie et si $\gamma$ est un point critique alors
$$Q_X(t) := \dr{L}{v^i}(t,\gamma,\dot\gamma)X^i(t,\gamma) - \left(\dr{L}{v^i}(t,\gamma,\dot\gamma)\dot\gamma^i - L(t,\gamma,\dot\gamma)\right)T(t,\gamma)$$
est conservé (i.e. $\frac{\d Q}{\d t}=0$).\\
\\
\underline{Remarque}: $Q_X = \dr{L}{v^i} + LT$\\ \\
Preuve du théorème:
$\forall \gamma \in \mathcal{C}^1(I,\mathcal{M})$\\
\begin{align*}
1) \;\mathrm{hypothese}\;\mathrm{de}\;\mathrm{symetrie}\;
&\iff \int_{t_0}^{t_1} L(t,\gamma_\epsilon,\dot\gamma_\epsilon) = \int_{t_0}^{t_1}L(t,\gamma,\dot\gamma) + \epsilon [LT]_{t_0}^{t_1} + \int_{t_0}^{t_1} \left(\dr{L}{x^i}\delta \gamma^i + \dr{L}{v^i}\delta\dot\gamma^i\right) +o(\epsilon)\\
&\iff \int_{t_0}^{t_1} \delta_X L(t,\gamma,\dot\gamma) \d t:=\int_{t_0}^{t_1} \left(\dr{L}{x^i}\delta \gamma^i + \dr{L}{v_i}\delta\dot \gamma^i + \frac{\d (LT)}{\d t}\right)\d t =0
\end{align*}
où $\delta_X L : I\times T\mathcal{M} \to \mathbb{R}$. 
Bref, ``symétrie $\implies \delta_X L = 0$".\\
\\
\underline{Exo}: construire $\delta_X L$ et montrer que ça marche...\\
\\
2) Montrons que $Q$ constant si (et seulement si) $\gamma$ est un point critique.
$$\frac{\delta L}{\delta \gamma^i} := \dr{L}{x^i}(t,\gamma,\dot\gamma) - \frac{\d}{\d t}(\dr{L}{v^i}(t,\gamma,\dot\gamma))$$
D'où EL $\iff \frac{\delta L}{\delta \gamma^i}=0$
\begin{align*}
\dr{L}{x^i}&= \frac{\delta L}{\delta \gamma^i} + \frac{\d}{\d t}(\dr{L}{v^i})\\
\frac{\d Q_X}{\d t}
&= \frac{\d}{\d t} (\dr{L}{v^i}\delta \gamma^i + LT)\\
&= \frac{\d}{\d t} (\dr{L}{v^i})\delta \gamma^i + \dr{L}{v^i} \delta\dot\gamma^i+\frac{\d}{\d t}(LT)\\
&= \dr{L}{x^i} \delta^i+\dr{L}{v^i}\delta \dot \gamma + \frac{\d (LT)}{\d t} - \!\!\!\!\!\!\!\!\!\!\!\!\!\!
\underset{\quad\quad\quad=0\;\mathrm{par}\;(\mathrm{E}.\scalebox{0.75}[1.0]{-}\mathrm{L}.)}{\cancel{\frac{\delta L}{\delta \gamma^i}\delta \gamma^i}}\\
&=0 \quad \mathrm{par} \; \mathrm{symetrie}
\end{align*}
\\
\underline{Variante}: Si $\exists f:I\times\mathcal{M}\to \mathbb{R}$ t.q. $\delta_X L(t,\gamma,\dot\gamma) = \frac{\partial f}{\partial t}(t,x) + v^i \frac{\partial f}{\partial x^i}(t,x)$ ``symétrie modulo un terme exacte" (def) alors la quantité conservée est $(Q_X -f)$.\\

Même si on va aller plus loin dans les théorèmes de Noether plus tard, une bonne référence (historique) est \emph{Les Théorèmes de Noether: Invariance et lois de conservation au XXe siècle} par  Yvette Kosmann-Schwarzbach, éditions de l'école Polytechnique, ISBN: 978-2730211383.


\subsection{Formalisme Hamiltonien}
L'idée est de faire un changement de variable de $T\mathcal{M}$ vers $T^*\mathcal{M}$... Commençons par définir un variété symplectique.
\\
\underline{Définition}: (variété symplectique)\\
Un var symplectique $\mathcal{M}$ est une variété munie d'une 2-forme $\omega$, $\omega \in \Omega^2(\mathcal{M})$ telle que:
\begin{itemize}
\item $\omega$ non dégénérée i.e. $\forall \xi\in T\mathcal{M}, \quad \quad \xi \lrcorner \omega = 0 \quad \implies \quad \xi =0$\\
$\xi \lrcorner \omega:= \omega(\xi,\cdot)$ (également noté, $\iota_\xi \omega$ dans d'autres ressources)
\item $\d \omega = 0$\quad\quad ``forme fermée"
\end{itemize}

Dans des coordonnées locales, $\omega = \sum_{1\leq a_1 < a_2\leq n} \omega_{a_1a_2} \d x^{a_1} \wedge \d x^{a_2}$, et les hypothèses reviennent à dire que le rang de la matrice $(w_{a_1a_2})$ est maximal, d'où dim$\mathcal{M}$ paire.\\
\\
\underline{Théorème de Darboux}:\\
Dans toute variété symplectique, tout point admet une carte (et un jeu de coordonnées $(p_i)\!\!\smile\!\!(q^i)$ sur cet ouvert) dans laquelle $\omega = \d p_i\wedge\d q^i$.\\
\\
Constructions ultra classiques de var symplectiques:\\
a) $\mathbb{R}^{2n}=\mathbb{R}^n\times\mathbb{R}^n$, on peut donc définir $\omega$ comme dans le théorème de Darboux sur tout $\mathbb{R}^{2n}$.
\\
b) Soit $\mathcal{M}$ une variété de dimension $n$, 
\begin{equation*}
\exists!\; \pi : \left\{\begin{matrix}
(T^*\mathcal{M}) & \to & \mathcal{M} \\
(q,p) & \mapsto & q
\end{matrix}\right.
\end{equation*}
Soit ce$\pi$ et soit $\theta \in \Omega^1 (T^*\mathcal{M})$ tel que,
$$\forall \xi \in T_{(p,q)}(T^*\mathcal{M}), \quad \theta_{(q,p)}(\xi) = \langle \underset{\in T_q^*\mathcal{M}}{p,} \underset{\in T_q\mathcal{M}}{\d \pi_{(q,p)} \xi} \rangle$$

En coordonnées locales, $(q^i)$ sur $\mathbb{M}$ et $(p_i)$ sur $T_q^*\mathcal{M}$ avec $p := p_i \d q^i$, on obtient $\theta = p_i \d \left(q^i\circ \pi\right)$ où $(p_i)\!\!\smile\!\!(q^i)$ sont des coordonnées locales sur $T^*\mathcal{M}$. On notera tout simplement $\theta = p_i\d q^i$ avec $\theta \in \Omega^1(T^*\mathcal{M})$
ce qui est un abus de notation conséquent (notamment puisque rentrant violemment en conflit avec la définition de $p$). Bref, il faut ouvrir l'œil au contexte.\\
\\
Il suffit alors de prendre $\omega := \d \theta$ forme symplectique, pour avoir $(T^*\mathcal{M},\omega)$ une variété symplectique.

\noindent \underline{Lien entre Lagrangien et géométrie symplectique} (eq° de Hamilton)\\
L'objectif est d'effectuer une transformation de la forme:
\begin{equation*}
L: \left\{ \begin{matrix}
\mathbb{R}\times T \mathcal{M} & \to & \mathbb{R}\\
(t,x,v) & \to & L(t,x,v)
\end{matrix}\right.
\quad \leftrightsquigarrow \quad H:\left\{\begin{matrix}
\mathbb{R}\times T^* \mathcal{M} & \to & \mathbb{R}\\
(t,q,p) & \to & H(t,q,p)
\end{matrix}\right.
\end{equation*}
Si $(\d q^i)$ est une base de $T_q^*\mathcal{M}$, $p\in T^*_q\mathcal{M} \implies p=p_i\d q^i$, d'où $(p_i, q^i)$ est un système de coordonnées sur $T^*\mathcal{M}$; enfin, en fait c'est $q^i\circ \pi$ à la place de $q^i$ mais bon, c'est l'abus de notation de tout à l'heure. On pose:
$$\theta = p_i\d q^i$$
\underline{Transformation de Legendre}:
$$\forall (t,q) \in \mathbb{R}\times\mathcal{M}\quad\quad \d \left(L_{|\{t\}\times T_q\mathcal{M}}\right) =: \dr{L}{v}(t,q,v)$$
en coordonnés locales, $v=v^i\frac{\partial}{\partial q^i} \in T_q\mathcal{M}$
$$\dr{L}{v} = \dr{L}{v^i} \d v^i$$
Hypothèse de Legendre: 
\begin{equation*}
\mathbb{L}: \left\{\begin{matrix}
\mathbb{R}\times T\mathcal{M} & \to & \mathbb{R}\times T^*\mathcal{M}\\
(t,q,v) & \mapsto & (t,q, \dr{L}{v}(t,q,v))
\end{matrix}\right. \quad \mathrm{est}\; \mathrm{un}\; \mathrm{diffeo}
\end{equation*}
Exemple: $L = \frac{m |v|^2}{2} - V(q)$\\
\\
\underline{Définition}: (Hamiltonien)
\begin{align*}
H : \mathbb{R}\times T^*\mathcal{M} &\to \mathbb{R}\\
(H\circ\mathbb{L})(t,q,v) &= \dr{L}{v^i}(t,q,v) v^i - L(t,q,v)\\
\iff (\mathrm{implicit})\quad  \mathbb{L}^{-1}&: (t,q,p)\to (t,q,v(t,q,p))
\end{align*}
$$\dr{L}{v^i}(t,q,v(t,q,p)) =: p_i $$
$$H(t,q,p) = p_i v^i(t,q,p) - L(t,q,v(t,q,p))$$

\subsection{retours sur Noether}
$T\frac{\partial}{\partial t} + X^i \frac{\partial}{\partial x^i}$ sur $\mathbb{R}\times\mathcal{M}$ est une symétrie de $L$.

$\implies Q = \dr{L}{v^i}(t,\gamma, \dot\gamma)X^i(t,\gamma) - (\dr{L}{v^i}\dot\gamma^i - L(t,\gamma,\dot\gamma))$ est conservé si $\gamma$ est solution.

$$Q = \underset{``\mathrm{moment}"}{(p_i \circ \mathbb{L})}X^i - \underset{``\mathrm{energie}"}{(H\circ\mathbb{L})}T$$
METTRE LES SOUSTITRES

\begin{align*}
\d H &= v^i \d p_i + \cancel{p_i\d v^i} - \dr{L}{t}(t,q,v) \d t - \dr{L}{q_i}\d q^i - \cancel{\dr{L}{v^i}(t,q,v(t,q,p) \d(v^i)}\\
&= v^i\d p_i - (\dr{L}{t}\circ \mathbb{L}^{-1})\d t - (\dr{L}{q^i}\circ \mathbb{L}^{-1}) \d q^i
\end{align*}
D'où 
\begin{align*}
	\dr{H}{t} &= -\dr{L}{t}\circ \mathbb{L}\\
	\dr{H}{q^i} &= - \dr{L}{x^i} \circ \mathbb{L}\\
	\dr{H}{p_i} &= v^i
\end{align*}
d'où $\forall \gamma:\mathbb{R} \to \mathcal{M}$
$$\pi = \dr{L}{v}(t,\gamma,\frac{\d \gamma}{\d t})$$\\

\noindent \underline{Lemme}: (transition Lagrangien-Hamiltonien)
\begin{equation*}
\frac{\d}{\d t} (\dr L{v^i} (t,\gamma, \dot \gamma)) = \dr L{q_i} (t,\gamma,\dot\gamma)\quad (\mathrm{EL}) \quad \iff \quad
\left\{\begin{split} \frac{\d \gamma^i}{\d t}&=\dr{H}{p_i}(t,\gamma,\pi) \quad (\mathrm{Hq})\\
\frac{\d \pi_i}{\d t}&=-\dr{H}{q^i}(t,\gamma,\pi) \quad (\mathrm{Hp})\end{split}\right.
\quad\quad\quad\quad\quad\quad\quad\quad\quad\quad\quad\quad\quad\quad\quad\quad\quad\quad\quad\quad\quad\quad\quad\quad\quad\quad\quad\quad\quad\quad\quad
\end{equation*}
Preuve:
$$Hq \iff (t\gamma,\pi) = \mathbb{L}(t,\gamma,\dot\gamma)$$
$$\dr{H}{p_i}(t,\gamma,\pi)=v^i(t,\gamma,\pi)=\frac{\d \gamma^i}{\d t}$$
par def de $v$.\\
Alors, $\pi_i = \dr{L}{v_i}(t,\gamma,\dot\gamma)$
$$\frac{\d \pi}{\d t} = \frac{\d}{d t}(\dr{L}{v_i}) = (EL) \dr{L}{x^i} = -\dr{H}{q^i}$$
Notation:
\begin{align*}
\frac{\d q^i}{\d t} &= \;\;\,\dr H{p_i}\;\\
\frac{\d p_i}{\d t} &= -\dr H{q^i}
\end{align*}

\subsection{Formulation Géométrique}
$t\mapsto (\gamma(t), \pi(t)) \in \mathcal{C}^1(\mathbb{R},T^*\mathcal{M})$ est solution de Hamiltion
$$\iff \frac{\d}{\d t}(\gamma^i, \pi_i) = (\dr H{p_i},-\dr H{q^i})(\gamma,\pi)$$
champ de vecteurs \color{red} non-autonome (i.e. indépendant de $t$) \color{black} tangent à $T^*\mathcal{M}$.
$$X_H = \dr H{p_i} \dr{}{q^i}-\dr H{q^i}\dr{}{p_i}$$
$$\omega = \d p_i \wedge \d q^i$$
\begin{align*}
X_H \lrcorner \omega
&= \left(\dr H{p_i} \dr{}{q^i}-\dr H{q^i}\dr{}{p_i}\right) \lrcorner \d p_j\d q^j\\
&= \dr H {p_i} (-\delta^{ij} \d p_j) - \dr H {q^i}(\delta_{ij}\d q^j)\\
&= - \left(\dr H {p_i} \d p_i + \dr H {q^i} \d q^i\right)\\
&=\dr H t - \d H
\end{align*}
bref:
$$\boxed{X_H \lrcorner \omega + \d H = \dr H t \d t}$$
Artifice: $T^*(\mathbb{R}\times\mathcal{M}) \supset (\mathbb{R}\times\{0\})\times T^*\mathcal{M}\approx\mathbb{R}\times T^*\mathcal{M} $
\\
On pose alors $q^0 = t$ et sur $T^*(\mathbb{R}\times\mathcal{M})$ on étends $\Tilde \omega:= \d p_0 \wedge \d t + \d p_i \wedge \d q^i$ d'où
$$X_{\Tilde H} \lrcorner \Tilde \omega + \d \Tilde H=0$$
et donc on s'intéresse uniquement à l'hyper-surface $p^0 = H$.

\section{Théorèmes de Noether généraux}
\subsection{Théorème 1}
Lagrangien d'ordre quelconque $r$, i.e. $L(x, \dot x, \ddot x, \dot{\ddot x}\;...\; x^{(r)})$. On travaille sur des champs $u:U\to\mathcal{M}$ où $U=\mathbb{R}$ dans le cas particules, mais sinon peut-être n'importe quoi (ligne d'univers d'une particule dans l'espace-temps, champ classique, ou des produits de ça...).
\\
\underline{Définition}: (Jets)\\
Si $\mathcal{M}$ est varr de dim $k$ et $U$ est un ouvert de $\mathbb{R}^n$, 
$$\mathfrak{j}^r u (x) := (x, u(x), \partial u(x), \partial^2 u(x), \,...\, \partial^r u (x)$$
où $\partial^i := \dr{}{^{\mu_1}...\partial^{\mu_r}} =: \partial_{\mu_1...\mu_r}$
Cas général pour des variétés quelconques:

\begin{align*}
	\mathfrak{j}^0(U,\mathcal{M}) &= U\times\mathcal{M}\\
	\mathfrak{j}^1(U,\mathcal{M}) &= \{(x,y,E), \quad (x,y)\in U\times\mathcal{M}, E \;\mathrm{sev}\;\mathrm{de}\; T_{(x,y)}(U\times\mathcal{M})\\&\quad \quad| \quad \mathrm{dim}E=\mathrm{dim}U\\
	&\quad\quad\quad\;\d (\pi_{U\times\mathcal{M}\to U})_{(x,y)}: T_{(x,y)}(U\times\mathcal{M})\to T_x\mathcal{M}
	\\&\quad\quad\quad\; \d (\pi_{U\times\mathcal{M}\to U})_{x,y)}|_E : E\to T_x\mathcal{U}\quad\quad\quad\quad\quad\quad\quad\quad\quad\}\\
	\mathfrak{j}^r(U,\mathcal{M}) &= \mathfrak{j}^1(U, j^{r-1}(U,\mathcal{M}))
\end{align*}
Système de coordonnées locales sur les jets:
$$v^i_{\mu_1...\mu_j} \quad\quad \mathrm{t}.\mathrm{q}. \quad\quad v^i_{\mu_1...\mu_j}(j^ru(x))=\dr{u^i}{x^{\mu_1}...\partial x^{\mu_j}}$$
Lagrangien général d'ordre $r$ sur ``$U\to\mathcal{M}$":
$$L : j^r(U,\mathcal{M}) \to \mathbb{R}$$
$$\mathcal{L}[u] = \int_{U} L(j^r u(x))\d^n x$$

Symétrie infinitésimale $u\mapsto u+\epsilon\delta u + o(\epsilon)$ infinitésimales, générés par un champ de vecteurs $Z$ sur $U\times\mathcal{M}$. Ou plutôt, pour être précis, un champ $Z:j^r(u)\to T(U\times\mathcal{M})$.
$$Z = X^\mu \partial_\mu + Y^i  \partial_i$$
$$\delta u^i = Y^i - \dr{u^i}{x^\mu}X^\mu$$
\underline{Théorème de Noether 1}: (Forme la plus générale)

Si $L$ est invariant par $X^\mu\partial_\mu + Y^i\partial_i$ et si $u$ est un point critique de $\mathcal{L}$ alors il lui correspond $J^\mu\partial_\mu$ définit sur $U$ tel que $\dr {J^\mu}{x^\mu}=0$\\
\\
Ex: $u: \mathbb{R}^n \to \mathbb{R}$ \quad\quad $\Omega\subset \mathbb{R}^n$
\quad \quad $\mathcal{L}[u]=\int_{\Omega} \frac{|\nabla u|^2}2 \d x$ (action de Dirichlet)\\
$\mathcal{L}[u+\epsilon\varphi] = \int_\omega \frac{|\nabla u|^2}2 + \epsilon \langle\nabla u, \nabla \varphi\rangle + \epsilon^2 \frac{|\nabla \varphi|^2}2$
 ($\varphi$ supposé à support compacte.)
\begin{align*}
\delta \mathcal{L}_u[\varphi] &= \int_\Omega \langle\nabla u, \nabla \varphi\rangle \d x\\
&= \int_\Omega \big(\mathrm{div}(\varphi\nabla u) - \varphi \Delta u\big) \d x\\
&=- \int \varphi \Delta u
\end{align*}
Symétrie par translation $u \mapsto u\circ \tau_\epsilon =: u_\epsilon$; $\tau_\epsilon (x) := x-\epsilon v$.\\
$u_\epsilon(x) = u(x-\epsilon v) \approx u(x) - \epsilon v^i \dr u {x^i}(x) + o(\epsilon)$\\
$\delta u = - v^i \dr u {x^i}$\\
Noether: Si $\delta u = 0$, 
$$\dr L {v_\mu} (x,u,\d u) \dr u {x^\nu} - (L(x,u, \d x) \delta_\mu^\nu) v^\mu = J^\nu$$
alors $\dr{J^\nu}{x^\nu} = 0$\\
(le prof est pas totalement sûr de la formule pour $J$, voir la démo qui suit)\\
\underline{Cas particulier}:
$r=1$, i.e. $L(x,u,\partial u)$, $X^\mu (x,u), Y^i(x,u)$
$$J^\mu = \dr L {v^i} (x,u,\partial u) Y^i - \left(\dr L {v^i_\mu} \dr{u^i}{x^\nu} - L \delta_\nu^\mu\right) X^\nu$$
et EL $\implies \dr{J^\mu}{x^\mu}=0$\\
\underline{Démonstration}: (cas général)\\
$X^\mu \partial_\mu + Y^i \partial_i$ agissant sur $(U,u)$.\\
$U\mapsto U_\epsilon=\varphi_\epsilon(U)$.\\
$\varphi_\epsilon := x + \epsilon X + o(\epsilon)$.
$u\mapsto u_\epsilon = u + \epsilon \delta u + o(\epsilon)$.
$$\delta u^i := Y^i - \dr{u^i}{x^\mu}X^\mu$$
Symétrie $\overset{\mathrm{def}}{\iff} \quad \forall U \forall u \quad \mathcal{L}_{U_\epsilon}[e_\epsilon] = \mathcal{L} + o(\epsilon)$\\
Petit lemme de calcul ($m$ multi-indice):
$$0 = \int_U \left[\sum_{|m|<r}\dr L{v^i_m}\big(\mathfrak{j}^r(u)\big)\dr{^m\delta u^i}{x^m}+\dr{}{x^\mu}\big(L(\mathfrak{j}^r(u)X^\mu\big) \right] \d^n x$$
autre petit lemme:\\
$\rho(\epsilon,x):=L\big(\mathfrak{j}^ru_\epsilon(x)\big)$
$$\frac{\d}{\d \epsilon}\left( \int_{\varphi_\epsilon(U)}\rho(\epsilon,x) \d  x \right) _{\Big|\epsilon=0} = \int_U \dr\varphi\epsilon(0,x)+\dr{}{x^\mu}\big(X^\mu\rho(0,x)\big)$$
et un dernier lemme:\\
Soit $A^{\mu_1 ... \mu_p}$ un tenseur symétrique, et $g$ une fonction sur $\Omega$. $(1\leq p\leq r)$
$$A^{\mu_1...\mu_p} \dr g{x^{\mu_1}...\partial x^{\mu_p}} = (-1)^p g \dr{A^{\mu_1...\mu_p}}{x^{\mu_1}...\partial x^{\mu_p}} + \dr{}{x^{\mu_1}}\left(A^{\mu_1...\mu_p}\overset{\leftrightarrow}\partial_{\mu_2...\mu_p}g\right)$$
où
\begin{align*}
f\overset{\leftrightarrow}\partial_{\mu_2...\mu_p} g:=
& f \partial_{\mu_2...\mu_p} g\\
&- \partial_{\mu_2} f \partial_{\mu_3...\mu_p} g\\
&+ ...\\
&+ (-1)^p (\partial_{\mu_2...\mu_p} f) g
\end{align*}
Tous ces lemmes se prouvent par du calcul un peu bourrin.\\
Ainsi, la condition de symétrie devient, via $A^m=\dr L{v^i_m}(\mathfrak{j}^r(u)$ et $g=\delta u^i$:
$$\mathrm{Symetrie} \iff \int_U \sum_{|m|<r} (-1)^{|m|} \dr{^{|m|}}{x^m}\left(\dr L{v^i_m}(\mathfrak{j}^r(u))\right)\delta u^i + \dr{}{x^\mu} \left(\sum_{|m|\leq r} \dr L{v^i_m}\overset{\leftrightarrow}\partial_{m\backslash\mu} \delta u^i + X^\mu L\right) = 0$$

\begin{align*}
&\mathrm{Posant}:\quad\quad\quad\quad\quad\quad\quad\quad&(\mathrm{EL})(u):=& \sum_{|m|\leq r} (-1)^{|m|} \dr{^{|m|}}{x^m}\left(\dr L{v^i_m}(\mathfrak{j}^r(u))\right)
\quad\quad\quad\quad\quad\quad\quad\quad\quad\quad\\
&\mathrm{et}&
J^\mu :=& LX^\mu + \sum_{|m|\leq r} \dr L{v^i_m} \overset{\leftrightarrow}\partial_{m\backslash\mu} \delta u^i
\end{align*}
On a bien
$$X^\mu \partial_\mu + Y^i \partial_i\quad\mathrm{Symetrie} \quad \quad \quad \iff\quad \quad \quad 
 \partial_\mu J^\mu = 0$$

\subsection{Théorème 2}
Hypothèse: il existe $X^{a,m,\mu}$ et $Y^{a,m,i}$ sur les jets tel que pour toute famille $(f_a)_{1\leq a \leq A}$ de fonctions $\mathcal{C}^\infty$ (ou $\mathcal{C}^{\mathrm{dim}\,\mathcal{M}}$) sur $\Omega\supset U$ on ait une (famille de) symétrie(s) via:
\begin{align*}
X^\mu =& \sum_a \sum_{|m|\leq r} X^{a,m,\mu}(\mathfrak{j}^r(u))\dr {f_a}{x^m}\\
Y^i =& \sum_a \sum_{|m|\leq r} Y^{a,m,i}(\mathfrak{j}^r(u))\dr {f_a}{x^m}
\end{align*}
\underline{Théorème de Noether 2}: (Cas des symétries de dimension infinie)\\
Si l'hypothèse ci-dessus est vérifiée, il y a dégénérescence de l'équation d'Euler-Lagrange.\\
\\
\underline{Démonstration}:
\begin{align*}
\delta u^i :=& Y^i - \partial_\mu u^i X^\mu\\
=& \sum_{|m|\leq r} \delta r ^{r,i} \partial_m f_a
\end{align*}
$$\mathrm{Symetrie} \quad \quad \iff \int_U (\mathrm{EL})(u)_i \delta u^i + \partial_\mu \left(\sum_{|m|\leq 2r-1} K^{a,\mu}\partial_m f_a\right)$$
a) on prend $\mathfrak{j}^{2r-1}f_a |{\partial u}=0$, d'où $\int_U (\mathrm{EL})(u)\delta u^i =0$\\
b) $(\mathrm{EL})(u)_i = \sum_m (-1)^r \partial_\mu (\dr L {v^i_m})$ 
$$(\mathrm{EL})(u)_i \delta u^i \quad = \quad (\mathrm{EL})(u)_i \sum_{|m|\leq r} \delta u^{m,i}\partial_m f_a \quad + \quad \partial_\mu \left(\sum_{|m|\leq r} (\mathrm{EL})\delta u^{m,a}\overset{\leftrightarrow}\partial_{m\backslash\mu} f_a\right)$$
Conclusion: $\forall f_a : \mathfrak{j}^{2r-1} f_a\,\!_{|\partial u} = 0$
$$\int \sum_{|m|\leq r} (-1)^{|m|}\partial_m\left[(\mathrm{EL})(u)_i (Y^{m,a} - \dr u{x^\nu} X^{m,a,\nu}\right] f_a = 0$$
\\
Exemple: Électromagnétisme\\
Rappel: Étoile de Hodge $* : \Omega^p(\mathcal{M})\to \Omega^{n-p}(\mathcal{M})$ pour passer de $J^\mu$ 3-forme à 1-forme...
\begin{align*}
\mathrm{Electromagnetisme} \quad \quad &\iff \quad \quad \left\{
\begin{matrix}\d F & = & 0\\ \d(*F) & = & J\end{matrix}\right.\\
& \iff \mathcal{A}[A] = \int \frac{1}{4}F_{\mu\nu}F^{\mu\nu} + A_\mu J^\mu \d^n x\\
& \quad \mathrm{avec} \quad F = \frac{1}{2} F_{\mu\nu}\d x^\mu \wedge \d x^\nu \quad \mathrm{et}\quad F_{\mu\nu}=\partial_\mu A_\nu - \partial_\nu A_\mu
\end{align*}
$A\mapsto A+\d \varphi,\quad \varphi\in \mathcal{C}^\infty_\mathrm{c}$ groupe de symétrie de Noether. D'où $J:=\d (*F) = \d (*\d A)$ est un problème sous-déterminé.
Autre exemple: (RG) $\mathcal{A}[g] = \int \mathrm{Ric}_g \d\mathrm{vol}_g$\\
(i.e. $R_{\mu\nu}-\frac{1}{2}Rg_{\mu\nu} = 0$) a ses symétries dans l'identité de Bianchi
$$\nabla_\mu \left(R^{\mu\nu}-\frac{1}{2}Rg^{\mu\nu}\right)=0$$
(et en fait, dans tous les difféomorphismes).

\section{Mécanique et Géométrie Symplectique}
\subsection{Vers une approche plus générale}
On rappel que:
$$\begin{matrix}
L: & \mathbb{R}\times T\mathcal{M} & \to & \mathbb{R}\\
\mathbb{L}: &  \mathbb{R}\times T\mathcal{M} & \to & \mathbb{R}\times T^*\mathcal{M}\\
& (t,x,v) &\mapsto &\left(t,x, p_i = \dr L{v^i}(t,x,v)\right)
\end{matrix}$$
Et avec l'hypothèse que $\mathbb{L}$ est un difféo, on construisait:
$$H(t,q,p):=p_iv^i(t,x,p)-L\big(t,x,v(t,x,p)\big)$$
avec $p_i:= \dr L{v_i}(t,x,v(t,x,p))$.\\
On obtenait alors les equations:
\begin{align*}
\frac{\d \gamma^i}{\d t} =& \quad \, \dr H{p_i}(t,\gamma, \pi)\\
\frac{\d \pi_i}{\d t} =& - \dr H{q^i}(t,\gamma, \pi)
\end{align*}
On obtenait alors un flot sur la variété symplectique $T^*\mathcal{M}$ par
$$X_H : \left\{\quad \begin{matrix}
0 & = & X_H \lrcorner + \d H\\
\omega & = & \d p_i \wedge \d q^i
\end{matrix}\right.$$
Mais on peut se ramener à des problèmes variationnels, en changeant un peu notre construction:\\
Nous allons maintenant travailler dans $T^*(\mathbb{R}\times\mathcal{M})$ au lieux de $\mathbb{R}\times T^*(\mathcal{M})$.
$$\mathcal{L}(\gamma,\zeta,\pi) := \int_I \color{red}\left[\color{black}L(t,\gamma,\zeta) \color{red}\cancel{\color{black}\d  t} + \pi \left( \frac{\d \gamma^i}{\d t} - \zeta^i\right)\right]\d t$$
\color{red}i.e. on impose $\zeta = \frac{\d \gamma}{\d t}$ via les multiplicateurs de Lagrange.\color{black}
$$\pi \mapsto \pi + \delta \pi \quad \quad \quad \rightsquigarrow \quad\quad\quad
\mathcal{L}(\gamma,\zeta,\pi) \mapsto \mathcal{L}(\gamma,\zeta,\pi) + \epsilon \int \delta\pi_i (\dot\gamma^i - \zeta^i)\d t$$
$$\forall\delta\pi \quad \quad \delta\mathcal{L}[\delta\pi]=0 \quad \quad \quad \iff\quad \quad \quad \zeta^i = \frac{\d\gamma^i}{\d t}$$
$$\delta\mathcal{L}[(0,\delta\zeta,0)] = \int_I \left(\dr L{v^i}\delta\zeta^i-\pi_i\delta\zeta^i\right)\d t = 0$$
i.e. :
\begin{align*}
\left\{\begin{matrix}
\gamma & \mapsto & \gamma\\
\pi & \mapsto & \pi\\
\zeta & \mapsto & \zeta + \epsilon\delta\zeta
\end{matrix}\right.
\iff & \pi_i = \dr L{v^i}\\
\iff & (t,\gamma,\pi) = \mathbb{L}(t,\gamma, \zeta)
\end{align*}
Alors:
\begin{align*}
\mathcal{L}[\gamma,\pi] &= \int_I \left[L\big(t,\gamma,v(t,\gamma,\pi)\big)
\right] \d t\\
&= \int_I \pi_i \dot \gamma^i - \left(\pi_i v^i(\gamma,\pi) - L\big(t,\gamma,v(t,\gamma,\pi)\right)\d t\\
&= \int_I \pi_i\dot\gamma^i - H(t,\gamma,\pi)\\
A[\pi,\gamma] &= \int_I \left(\pi_i \frac{\d \gamma^i}{\d t}- H(t,\gamma,\pi)\right) \d t
\end{align*}
A pour point critique les solutions de l'équation de Hamilton. (proof left as exo)\\
On appel cela l'\underline{action de Poincaré}.

\subsection{Trajectoires dans l'espace-temps}
On travaille donc dans $T^*(I\times \mathcal{M})$. On a des coordonnées dans $T^*\mathcal{M}$ via $(q^i,p_i)$, et on complète par $q^0:=t$ et $p_0$ son dual, pour faire $(q_\mu,p^\mu)$ coordonnées pour $T^*(I\times \mathcal{M})$.
$$\omega = \d p_0 \wedge \d q^0 + \d p_i \wedge \d p^i = \d p_\mu \wedge \d q^\mu$$
$$\mathcal{H}(p_\mu, q^\mu):= p_0 + H(q^0,q^1,p_i)$$
$$\mathcal{H}: T^*(I\times\mathcal{M}) \to \mathbb{R}$$
On construit également:
$$(\gamma,\pi) \mapsto \Gamma:=\Bigg\{\bigg(t,\quad \gamma^i(t) - H\Big(t,\gamma(t),\pi(t)\Big),\quad\pi(t)\bigg),\; t\in I\Bigg\} \quad \quad \subset \mathcal{H}^{-1}(\{0\}) =: \mathcal{N}$$
$$\mathcal{L}[\gamma,\pi] = \int_\Gamma \underset{=:\theta}{p_\mu \d q^\mu}$$
$$\Gamma \quad \subset \quad \mathcal{N} \quad \subset \quad T^*(I\times\mathcal{M})$$

Notons qu'on se rapproche d'une description relativiste du mouvement (même si c'est pas encore tout à fait ça, car $\Gamma$ est toujours défini à travers notre choix de coordonnés initial dans $\mathcal{M}$).
On remplace $I\times\mathcal{M}$ pas une variété $\mathcal{E}$ (idéalement avec une métrique pseudo-Riemannienne, pour avoir un bon $*$). On a donc $\mathcal{H}:T^*\mathcal{E}\to\mathbb{R}$ et la dynamique est donnée par $\omega|_\mathcal{N}$. Explicitons...
$H$ sur $\mathcal{M}$ symplectique. Via le flot de $X_H$ on a:
$$X_H\lrcorner\omega+\d H = 0$$
$\mathcal{N}$ est une hyper-surface, telle que
$$\d\left(\omega_{|\mathcal{N}}\right)=0 \quad \quad \mathrm{et} \quad \quad
\omega_{|\mathcal{N}}=\mathfrak{i}_\mathcal{N}^*\omega$$
Rappel: $\d(\cdot)$ commute avec les pull-backs. \quad \quad \quad $\mathfrak{i}_\mathcal{N}\to T^*\mathcal{E}$
Notons que si $\omega_{|\mathcal{N}}$ est bien fermée, elle est par contre dégénéré (ainsi, ce n'est pas une forme symplectique sur $\mathcal{N}$).\\
ker $\omega_{|\mathcal{N}}$ = droite $\subset T\mathcal{N}$, qui décrit la dynamique.

\noindent \underline{Lemme}:\\
Soit $V$ un espace vectoriel de dimension finie:
$$V \quad \quad \supset \quad \quad W:= \mathrm{ker}\; (\alpha_1, \; ...\alpha_k) \quad \quad \quad \alpha_j \in V^*$$
\begin{align*}
V^* & \to W^* &&\\
\beta & \mapsto \beta_{|W} &&\\
\,& &&\\
V^*/\mathbb{R}(\alpha_i)_{i\in[\![1,k]\!]} & \to W^* &&\\
\quad\quad\quad\quad\quad\quad\quad\quad\quad\quad
\beta \mathrm{mod} [\alpha_1, \; ... \alpha_k] & \mapsto \beta_{|W}&&
\mathrm{est} \; \mathrm{un}\mathrm{iso}!
\end{align*}
Soit $(\mathcal{M},\omega)$ une variété symplectique, $T^*\mathcal{E}$, $\mathcal{N}\subset\mathcal{M}$, $M\in\mathcal{N}$, $X\in T_M\mathcal{M}$.
$$X\lrcorner \omega \in T_M^*\mathcal{M} \to X\lrcorner \omega_{|\mathcal{N}} \in T_M^*\mathcal{N}$$
Comme ker $\d\mathcal{H}=T_m\mathcal{N}$
\begin{align*}
\Big(X\lrcorner(\omega_{|T_M\mathcal{N}})=\Big) \quad \quad X\lrcorner \omega_{|T_M\mathcal{N}} = 0 \quad \quad &\iff \quad \quad X\lrcorner \omega \in \mathbb{R}\d \mathcal{H}\\
&\iff \exists \lambda \in \mathbb{R} \quad X = X_H \quad \mathrm{avec} \quad X_H\lrcorner\omega+\d \mathcal{H}=0
\end{align*}
$$\mathrm{ker}\left(\omega_{|T_M\mathcal{N}}\right)
:= \left\{X\in T_M\mathcal{N} \quad |\quad X\lrcorner\omega_{|T_M\mathcal{N}} = 0 \right\} = \mathbb{R}X_\mathcal{H}$$

On dit de $\left(\mathcal{N},\omega_{|\mathcal{N}}\right)$ que c'est une variété \underline{pré-symplectique} i.e. munie d'une forme fermée et de dégénérescence pas forcement nulle mais de noyau tangent à la dynamique.

Les courbes dans $\mathcal{N}= \mathcal{H}^{-1}(C)$ seront les points critiques de $\int_\Gamma \theta = \color{red} ?????\color{black}$, courbe tangente à la distribution ker $\omega_{|\mathcal{N}}$.

\noindent Autre exemple: (Force de Lorentz)
$$\mathcal{H} = (p_0 - eA_0)^2 - c^2|p_i-eA_i|^2_{\mathbb{R}^3} - (mc^2)^2$$

\subsection{Lien avec le premier théorème de Noether}
Situation:
$$\gamma : \left\{ \begin{matrix}
I & \to & \mathcal{M}\\
t & \mapsto & \gamma(t)
\end{matrix}\right.
\quad \quad \quad \quad
L[\gamma]=\int_I L(t,\gamma,\dot\gamma)\d t$$
$$X^i(t,x)\dr{}{x^i} + T(t,x)\dr{}t \quad \in \quad (\underset t T \times \underset x {\mathcal{M}})$$
est une symétrie (modulo $\d f$) de $L$ si 
$$\quad \quad \quad T\dr L t + \left(L - v^i \dr L {v^i}\right)\left(\dr T t + v^i \dr T {x^i}\right) + X^iL + \dr L{v^i} \left(\dr {x^i} t + v^j \dr{X^i}{x^i}\right)
= \dr f t + v^i \d f {x^i} \quad \quad \quad (*)$$
où $f: I\times\mathcal{M}\to \mathbb{R}$
\begin{align*}
\iff \quad \mathrm{Si}\; \mathcal{H} &= p_0 + H(t,q,p)\\
F &= p_0 T(q^0,q^i) + p_i X^i(q^\mu) - f(q^\mu)\\
&= \theta(T,X) - f, \quad\quad\quad\quad\quad\quad\quad\quad\quad\quad \theta=p_\mu\d q^\mu
\end{align*}
Or, si $f,g\in \mathcal{C}^\infty\big(T^*(I\times\mathcal{M})\big)$
$$\{f,g\} := \dr f{p_\mu}\dr g{x^\mu} - \dr f{x^\mu}\dr g{p_\mu}$$
\begin{align*}
(*) \iff \quad \quad \;\{H,F\} =& H\{H,T\}\\
\underset {\mathrm{si}\; \mathcal{N}=\mathcal{H}^{-1}(0)} \implies \{H,F\}_{|\mathcal{N}} =& 0
\end{align*}
\underline{Point de vue ``Relativiste"}:\\
$\mathcal{E}$ espace-temps,\\
$\mathcal{H}: T^*\mathcal{E}\to\mathbb{R}$ fonction ``cohérente",\\
$\mathcal{N}=\mathcal{H}^{-1}(\{0\})$,\\
une courbe $\Gamma$ par point critique:
$$\int_{\Gamma\subset\mathcal{N}} \theta \quad \quad \to \quad \quad \Gamma \; \mathrm{t}.\mathrm{q}. \;\forall X\in T_M\Gamma\quad X\lrcorner(\omega_{|\mathcal{N}}) = 0$$
Si $F=\theta(X) - f = p_\mu X^\mu (q) - f(q)$ où $f\in\mathcal{C}^\infty$, $X^\mu$ est une symétrie (modulo $\d f$) lorsque $\{H,F\}_{|\mathcal{N}}=0$.

\noindent \underline{Point de vue non-relativiste}:\\
$H:T^*\mathcal{M}\to \mathbb{R}$, $H$ indépendant du temps. $X=X^i(x)\partial_i \in \mathfrak{X}(\mathcal{M})$ est une symétrie de $\int_I L(\gamma,\dot\gamma)\d t$ ssi $\{H, p_i X^i(q)\}=0$.\\
\underline{Généralisation plus générale}:sur une variété symplectique quelconque $\mathcal{M}$.\\
\\
\underline{Définition}: (crochet de poisson sur une variété symplectique)\\
$$\begin{matrix}
\mathcal{C}^\infty(\mathcal{M})\times\mathcal{C}^\infty(\mathcal{M}) &
\to & \mathcal{C}^\infty(\mathcal{M})&\\
(F,G) & \mapsto & \{F,G\}&
:= \omega(X_F,X_G)
\end{matrix}$$
Remarque, dans un jeu de coordonnées à la Darboux, ça donne:
$$\{F,G\} = \dr F{p_i}\dr G{q^i} - \dr F{q^i}\dr G{p_i}$$
Soit: $(\gamma,\pi): I \to \mathcal{M}$ t.q.:
\begin{align*}
\frac{\d (\gamma,\pi)}{\d t} =& X_H(\gamma,\pi)\\
\forall F \in\mathcal{C}^1(\mathcal{M}) \frac{\d F(\gamma,\pi)}{\d t} =&
\dr F{q^i}(\gamma,\pi)\frac{\d \gamma^i}{\d t} + \dr F{p_i}\frac{\d \pi_i}{\d t}\\
=& \{F,H\}(\gamma,\pi)
\end{align*}
Notons, au passage, les propriétés triviales:
$$\forall A,B,C \quad \quad \quad \{A,B\}=-\{B,A,\} \quad \quad \{AB,C\}=A\{B,C\}+\{A,C\}B$$
\underline{Théorème de Noether 1 dans le cas symplectique}:\\
Si $X_F$ est une symétrie de $H$ alors $F$ est conservé le long du flot de $X_H$.
\begin{itemize}
\item $X_F$ symétrie de $H \iff \d H(X_F)=0 \iff X_F\lrcorner\d H = 0$.
\item $F$ conservé le long du flot de $X_H$: $\d F(X_H) = X_H\lrcorner\d F = 0$
\end{itemize}
Preuve:
\begin{align*}
\{H,F\} :=& \omega(X_H, X_F)\\
=& (X_H\lrcorner \omega)(X_F)\\
=& -\d H(X_F) = - X_F \lrcorner \d H\\
=& X_H \lrcorner \d F
\end{align*}
$$\boxed{X_H\lrcorner \d F = - X_F \lrcorner\d H =\{H,F\}}$$
\begin{align*}
u : I \to& (\mathcal{M},\omega) &\frac{\d F(\omega)}{\d t} =& \d F u\left(\frac{\d u}{\d t}\right)\\
&&=& \d F u(X_H)\\
\frac{\d u}{\d t} =& X_H(\omega)
& =& \{H,F\}(u)
\end{align*}
\underline{Proposition}:\\
$F\mapsto X_F$ symétrie infinitésimale de $\omega$ implique
$$L_{X_F}\omega = X_F \lrcorner\d \omega + \d \underset{=-\d F}{\left(X_F\lrcorner \omega\right)}=0 - \d(\d F) = 0$$
Se pose la question de si cette proposition admet une réciproque...\\
Soit $X\in \mathfrak{X}(\mathcal{M})$ t.q. $L_X(\omega)=0$
$$0=L_X(\omega)=0+\d (X\lrcorner\omega)$$
d'où $X\lrcorner\omega$ est fermé.\\
En fait la réciproque dépend de la cohomologie de la variété:
$$H^1(\mathcal{M})=\{0\} \quad \implies \quad \exists F : X\lrcorner\omega = - \d F, \; \mathrm{i}.\mathrm{e}. \; X=X_F$$
Sinon, on peut dire que c'est localement vrai, mais c'est pas aussi fort évidement. Bref:\\
Si $H^1(\mathcal{M})=\{0\}$, $X$ est une symétrie physique si et seulement si $L_X\omega = 0 = L_XH=X\lrcorner\d H$.\\
\\
Premier lemme sympa: $X_{\{f,g\}}=[X_f,X_g]$ i.e.
$$\left(\mathcal{C}^\infty(\mathcal{M}), \{\cdot,\cdot\}\right) \quad \overset{X_{(\cdot)}}{\underset{\mathrm{morphisme}\;\d '\mathrm{algebre}\;\mathrm{de}\;\mathrm{Lie}}{\xrightarrow{\hspace*{4cm}}}} \quad \left(\mathfrak{X}(\mathcal{M}),[\cdot,\cdot]\right)$$
Preuve:\\
Montrons que $\d \{f,g\} + [X_f,X_g]\lrcorner\omega = 0$
\begin{align*}
\d \{f,g\} =& \d \left(X_f \lrcorner\d g\right)\\
=& \d \left(X_f\lrcorner\d g\right)+ X_f \lrcorner\underset{=0}{\d(\d g)}\\
\underset{\mathrm{DD}}=\!\!&\;L_{X_f}(\d g)\\
\underset{\mathrm{Leibneitz}}=\!\!\!\!\!\!\!\!&\quad L_{X_f}(-X_g\lrcorner\omega)\\
=&-\underset{=[X_f,X_g]}{L_{X_f}(X_g)}\lrcorner\omega-X_g\lrcorner \underset{=0}{L_{X_f}}\omega
\end{align*}
Deuxième lemme: $\{f,\{g,h\}\}+\{g,\{h,f\}\}+\{h,\{f,g\}\}=0$\\
Preuve:
\begin{align*}
0\quad&\!\!\!\!\!\!\!\underset{\mathrm{Lemme}\;1}= \Big([X_f,X_g]-X_{\{f,g\}}\Big)\lrcorner\d h\\
&= X_f \cdot (X_g\cdot h) - X_g \cdot (X_f \cdot h) - \{\{f,g\},h\}\\
&= \{f,\{g,h\}\} - \{g,\{f,g\}\}+\{h,\{f,g\}\} 
\end{align*}
\section{Variétés de Poisson}
\subsection{Introduction aux variétés de Poisson}
\underline{Définition}: (variété de Poisson)\\
variété $\mathcal{M}$ munie d'un crochet de Poisson
$$\{\cdot,\cdot\}: \begin{matrix}
\mathcal{C}^\infty(\mathcal{M})\times\mathcal{C}^\infty(\mathcal{M}) & \to & \mathcal{C}^\infty(\mathcal{M})\\
(F,G) & \mapsto & \{F,G\}
\end{matrix}$$
Vérifiant:
\begin{itemize}
\item Bilinéarité
\item anti-symétrie
\item identité de Jacobi (donc c'est un crochet de Lie)
\item Leibnitz
\end{itemize}

\noindent Lemme final: ($\sim$Darboux pour les variétés de poisson)\\
Dans tout système de coordonnées locales $(x_i)$,
$$\exists \pi = \sum_{i<j} \pi^{ij}(x) \partial_i \wedge\partial_j$$
$\partial_i\wedge\partial_j:=(\partial_i\otimes\partial_j-\partial_j\otimes\partial_i)$, d'où $\pi = \pi^{ij}\partial_i\otimes\partial_j$ une fois anti-symétrisé, de sorte que:
$$\{f,g\} = \sum_{ij}\pi^{ij} \partial_i (f) \partial_j (g)$$
$$\pi^{ab}\partial_b\pi^{a'a''} + \pi^{a'b}\partial_b \partial^{a''a}+\pi^{a''b}\partial_b\pi^{aa'}=0 \quad \quad(\mathrm{Jacobi})$$
$$\pi \in \Gamma(\mathcal{M},\lambda^2 T\mathcal{M})$$
Note: si on étends la dérivée de Lie au crochet de Schouten ($\sim$ dérivée de Lie sur les structures supérieures) alors $[\pi,\pi]=0$.
\\ \\
\underline{Exemple}: Dual d'une algèbre de Lie.
\subsection{Aparté sur les Algèbres de Lie}
Rappels de base: définitions équivalentes de l'algèbre de Lie canoniquement associée à un groupe de Lie $G$.
\begin{enumerate}
\item $\mathrm{Lie}(G) = T_e G$
\item $\mathrm{Lie}(G) = \{$ Champs vectoriels tangeants à $G$ invariants à gauche (resp à droite) par l'action du groupe sur lui-même$\}$.
\end{enumerate}
Autre rappel (de pure géo-diff):
$$(\varphi_* X) (x) = \d \varphi _{\varphi^{-1}(x)} (X(\varphi^{-1}(x))) \quad \quad \quad \varphi \; \mathrm{diffeomorphisme}$$
$$\varphi_* X \lrcorner \varphi^* \alpha = (X\lrcorner \alpha) \cdot \varphi \quad \quad \quad \mathrm{dualite}\;\mathrm{push}\!-\!\mathrm{forward}\;\&\;\mathrm{pullback}$$
Encore un rappel: $G$ groupe de Lie $\implies G \approx G' \subset \mathrm{GL}_N(\mathbb{R})$\\
On vas donc écrire l'action à gauche simplement: $L_g x =: g x$.\\ \\
Dernier Rappel: $X,Y$ invariants $\implies [X,Y]_{\mathfrak{X}(G)}$ invariant, d'où
$$[X,Y]_{\mathfrak{X}(G)} (e) =: [X,Y]_\mathfrak{g}$$

\noindent \underline{Point de vue dual}: (Forme de Mauer-Cartan)
$$\begin{matrix}
\mathfrak{g} & \to & \mathfrak{X}(G)\\
\xi & \mapsto & \begin{matrix}
\mathrm{le}\;\mathrm{champ}\;\mathrm{de}\;\mathrm{vecteurs}\\
\mathrm{invariant}\;\mathrm{qui}\;\mathrm{vaut}\;\xi\;\mathrm{en}\;e_G
\end{matrix}
\end{matrix}$$
est un morphisme d'algèbres de Lie, et $\tilde \xi (x) = x \cdot \xi$.\\
On en déduit un isomorphisme $\alpha_x : T_x G \to \mathfrak{g}$ (enfin, une application inverser en fait):
$$\alpha_x (x.\xi)= \xi \quad \quad \rightsquigarrow \quad\quad \alpha \in \Omega^1 (G\cdot \mathfrak{g}) = \Omega^1(G)\otimes \mathfrak{g}$$
C'est la \emph{forme de Mauer-Cartan}.
\\
\underline{Lemme}: (Mauer-Cartan ou Formule de Cartan)
$$\d \alpha + \frac{1}{2} [\alpha \wedge \alpha] = 0$$
où $\forall \alpha, \beta \in \Omega^1 \otimes \mathfrak{g}$:
\begin{align*}
[\alpha \wedge \beta ] (v,w) :=& [\alpha(v),\beta(w)] - [\alpha(w),\beta(v)]\\ =& [\beta\wedge\alpha](v,w)
\end{align*}
Considérant une base $(E_i)$ de $\mathfrak{g}$:
\begin{align*}
\alpha =& \alpha^i E_i, &\beta =& \beta^i E_i, &\alpha^i =& \alpha^i_\mu \d x^\mu, &\beta^i =& \beta^i_\mu \d x^\mu
\end{align*}
\begin{align*}
[\alpha\wedge\beta] =& \big[(\alpha^i E_i)\wedge(\beta^jE_j)\big]\\
=& \alpha^i\wedge\beta^j [E_i,E_j]\\
=& C^k_{ij} \alpha^i \wedge \beta^j E_k
\end{align*}
avec $C^k_{ij}$ les coefficients de structure de l'algèbre de Lie dans $\mathfrak{g}$ pour la base $(E_i)$. Bref:
$$[\alpha\wedge\beta]^k = C^k_{ij} \alpha^i \wedge \beta^j$$
\underline{Preuve}: (formule de Cartan)
$$\d \alpha (X,Y) + \alpha\big([X,Y]\big) = X\dot \alpha(Y) - Y\cdot \alpha(Y)$$
\begin{align*}
X &= x\cdot \xi & Y&= x \cdot \zeta & (\xi,\zeta)&\in \mathfrak{g}
\end{align*}
\begin{align*}
\d \alpha_x (x\cdot \xi, x \cdot \zeta) + \alpha_x \big([x\cdot \xi, x\cdot \zeta]\big) &= x \cot [\xi, \zeta]\\
&= (x\cdot\xi) \lrcorner \underset{=0}{\d\alpha_x(\underset{=\zeta}{x\cdot\zeta})} - x\cdot \zeta \lrcorner \underset{=0}{\d (\xi)}\\
\alpha_x\big(x\cdot[\xi,\zeta]\big) &= [\xi,\zeta]\\
&= \big[\alpha_x (x\cdot\xi), \alpha_x(x\cdot \zeta)\big]\\
&=\frac{1}{2} [\alpha\wedge\alpha] (x\cdot \xi, x \cdot \zeta)
\end{align*}
Bref,
$$\boxed{\left(\d\alpha + \frac{1}{2}[\alpha\wedge\alpha]\right)(x\cdot\xi,x\cdot\zeta) = 0}$$
\underline{Retour à Poisson}: (Duale d'une algèbre de Lie comme exemple non-trivial de variété de Poisson)\\
$\mathfrak{g}$ algèbre de Lie, $(E_i)$ base de $\mathfrak{g}$, $C^k_{ij}:=[E_i,E_j]^k$ coefficients de structure, $\{\cdot,\cdot\}$ sur $\mathcal{C}^\infty(\mathfrak{g}^*)^2$.\\
$\forall F,G\in\mathcal{C}^\infty(\mathfrak{g}^*), \forall\alpha\in \mathfrak{g}^*$ \quad 
$\d F_\alpha \in T_\alpha^* (\mathfrak{g}^*) \approx (\mathfrak{g}^*)^* \approx \mathfrak{g}$, et de même, $\d G_\alpha \in \mathfrak{g}$.\\
On pose donc:
$$\{F,G\}(\alpha):=\langle\underset{\quad\in \mathfrak{g}^*}\alpha,\underset{\in\mathfrak{g}}{[\d F_\alpha,\d G_\alpha]}\quad \rangle\!\!\!\underset{\mathrm{crochet}\;\mathrm{de}\;\mathrm{dualite}}\; \in \mathcal{C}^\infty(\mathfrak{g}^*)$$
Il est trivial que ce crochet est bilinéaire, antisymétrique, Jacobi se vérifier simplement (c'est $\langle\alpha,\cdot\rangle$ qui contient toute cette structure), quand à Leibniz, on l'obtient directement en passant en coordonnées via:
$$\{F,G\} (\alpha) = \alpha_i C^i_{jk}\dr F{\alpha^j}(\alpha)\dr G{\alpha^k}(\alpha)$$
Réciproquement: si $V$ est un espace vectoriel, et $\{\cdot,\cdot\}$ est un crochet de Poisson sur $V^*$ linéaire, alors $V$ est une algèbre de Lie.\\
En gros, tout se trouve dans la dualité: $\pi_{ij}(\alpha) = C^k_{ij}\alpha_k$

\subsection{Retour à Poisson}
\noindent \underline{Lien avec Noether}: (application moment - Souriau)\\
Dynamique dans $(\mathcal{M},\pi)$ variété de Poisson.
$$\mathcal{C}^\infty(\mathcal{M}) \ni H \mapsto X_H \quad \quad \mathrm{t}.\mathrm{q}. \quad\quad \forall F \in \mathcal{C}^\infty(\mathcal{M}) \quad X_H \lrcorner \d F = \{H,F\}$$
i.e. $X_H$ agit comme un opérateur différentiel d'ordre 1.
$$\{H,F\}(x) = \pi^{ij}(x) \dr H{x^i} \dr F{x^j} \quad \quad \quad \rightsquigarrow \quad \boxed{X_H = \pi^{ij}(x) \dr H{x^i} \dr{}{x^j}}$$
\underline{Équations ``de" Hamilton}:\\
Pour $\gamma: I\to \mathcal{M}$, 
$$\frac{\d\gamma}{\d t} = X_H(\gamma)$$
$$\frac{\d F(\gamma)}{\d t} = \{H,F\}(\gamma)$$
Maintenant, supposons qu'il existe $G$, groupe de Lie, qui agit sur $(\mathcal{M},\pi)$ en respectant $\pi$ (i.e. une action à droite laissant la dynamique invariante).\\
On rappel les propriétés élémentaires de l'exponentielle:
$$\begin{matrix}
\mathfrak{g} & \to & G\\
\xi & \mapsto & \e^\xi
\end{matrix}
\quad \quad \quad
\frac{\d \left(\e^{t\xi}\right)}{\d t} = \e^{t\xi}\cdot\xi
 \quad \quad \quad
\e^{t\xi}_{|t=0} = e_G$$
elle induit une action de $\mathfrak{g}$ sur $\mathcal{M}$.\\
Hypothèses:
$$\Psi: \left\{
\begin{matrix}
\mathfrak{g} & \to & \mathfrak{X}(\mathcal{M})\\
[\cdot,\cdot]_\mathfrak{g} & \mapsto & \{\cdot,\cdot\}
\end{matrix}\right.\quad \mathrm{morphisme}$$
et $\forall \xi$ $\Psi(\xi)$ satisfait:
\begin{itemize}
\item Symplectique: $\Psi(\xi) \lrcorner \omega + \d \big((H,\xi)\big) = 0$
\item Poisson: $\d F\big(\Psi(\xi)\big) = \{(J,\xi),F\}$
\end{itemize}
où $J\in\mathcal{C}^\infty(\mathcal{M},\mathfrak{g}^*)$= ``application moment".

\noindent \underline{Noether symplectique}:\\
si $\Psi(\xi)\lrcorner \d H = 0$ et que $\frac{\d \gamma}{\d t} = X_H(\gamma)$ alors $J(\gamma)$ est constant.\\
Preuve:\\
$$\forall\xi \quad \frac{\d \left(\langle J,\xi\rangle(\gamma)\right)}{\d t} = \d \langle J,\xi\rangle_\gamma \!\!\!\!\!\!\!\!\!\!\!\underset{\quad\quad\;\;=X_H(\gamma)}{(\dot \gamma)} = \big\{H,\langle J,\xi\rangle\big\}(\gamma)$$
\underline{Exemple Physique}: Problème à deux corps\\ \\
\underline{Exemple ``canonique"}: $T^* G$ variété symplectique\\
(sans preuves, mais voir les notes pour détails)
\begin{enumerate}
\item Action à droite de $G$ sur $T^*G$:
$$\forall g \in G \quad R_g :\left\{
\begin{matrix}
G & \to & G\\
x & \mapsto & xg
\end{matrix}
\right.
\quad\quad\quad\quad\quad
\tilde R_g : \left\{
\begin{matrix}
T^*G & \to & T^*G\\
(x,a) \bigg\{\begin{matrix}
x \in G\quad\\
a \in T^*_x G
\end{matrix}
& \mapsto & \tilde R_g (x,a)
\end{matrix}
\right.$$
$$\tilde R_g (x,a) = \left(R_g x, R^*_{g^{-1}} a\right)= \left(x,g, a\circ \d R_{g^{-1}}\right)
\quad \quad \quad \quad\quad\quad
\tilde R_{g_1,g_2}:= \tilde R_{g_1}\circ \tilde R_{g_2}$$
\item $g\mapsto \e^{t\xi}$, $\xi\in \mathfrak{g}$ champ de vecteur invariant à droite.
$$X_\xi = \frac{\d \tilde R_{\e^{t\xi}}}{\d t}_{|t=0}
\quad \quad \quad \quad
X_\xi(x,a) = \bigg(x\cdot \xi, -(\mathrm{ad}_\xi p ^*) a \bigg)$$
où $p$ est donné par:
\item $X_\xi$ est une action hamiltonienne; $\sigma$ la forme symplectique usuelle sur $T^*G$. Or $\exists! p$ t.q.
$$p: \begin{matrix}
T^*G & \to & \mathfrak{g}^*\\
(x,a) & \mapsto & p(x,a)
\end{matrix} \quad \quad
a = p_i(x,a) \alpha^i(x)$$
où $\alpha\in \Omega^1$ est la forme de Mauer-Cartan. 
i.e. $\exists! p \; : \; \langle p(x,a), \alpha_x\rangle = a$.
Notons, au passage, que $\alpha = x^{-1}\d x$ en notation matricielle.

\noindent Ainsi, $$\boxed{X_\xi \lrcorner \sigma + \d \langle p, \xi \rangle = 0}$$
et aussi:
$$\forall f, g \in \mathcal{C}^\infty (\mathfrak{g}^*), \quad \quad
\boxed{\{f\circ p, g\circ p\}_{T^*G} = \{f,g\}_{\mathfrak{g^*}}\circ p}$$
\end{enumerate}
on dit que $p$ est un ``\underline{morphisme de Poisson}".
\subsection{Poisson, distributions et feuilletages}
Soit $(\mathcal{M},\pi)$ une variété de Poisson.\\
\underline{Motivation via exemple}: $\mathcal{M}=\mathfrak{so}(3)^*\approx \mathbb{R}^3$; $\mathfrak{so}(3) = \mathrm{Vect}\left(x^i\dr{}{x^{i+1}}-x^{i+1}\dr{}{x^i}\right)_{i\in \mathbb{Z}/2g\mathbb{Z}}$\\
On étudie la distribution:
$$D_x := \big\{ X_F(x), F\in \mathcal{C}^\infty(\mathfrak{so}(3)^*)\big\}=: x^\perp\subset T_x \mathfrak{so}(3)^* = \mathfrak{so}(3)^* \quad \mathrm{car} \; \mathrm{espace}\; \mathrm{vectoriel}$$
$D_x$ est une distribution singulière (singularité en 0)
$$(D_x)_{x\in\mathfrak{so}(3)^*}=\big\{(x,x^\perp),x\in \d \mathbb{R}^3\backslash\{0\}\big\} \cup \{(0,0)\}$$
est la distribution tangente aux sphères.

\noindent\underline{Cas général}: $(\mathcal{M},\pi)$
$$\forall x\in \mathcal{M} \quad D_x = \{X_F(x), F\in \mathcal{C}^\infty(\mathcal{M})\subset T_x\mathcal{M}\}$$
Si $\mathcal{M}$ est symplectique, $D_x = T_x \mathcal{M}$\\

\noindent\underline{Proposition 1}:\\
Si le rang de $D$ est constant, comme $X_{\{F,G\}}=[X_F,X_G]$. Soit $F_1, ..., F_k$ t.q. $X_{F_1}, ... X_{F_k}$ base de $D_x$.\\
$[X_{F_i},X_{F_j}]\in \mathrm{Vect}(X_{F_1},...X_{F_k})$\\

\noindent\underline{Théorème}: (Frobenius)\\
Si le rang de $D$ est constant, $D$ est intégrable.\\
D'où $\mathcal{M}$ feuilleté par des sous-variétés intégrable de $\mathcal{M}$.\\

\noindent Soit $\mathcal{F}$ une feuille intégrable
\begin{enumerate}
\item Si $\varphi \in \mathcal{C}^\infty(\mathcal{M}) \quad\quad\quad\quad\quad\quad\quad\quad\quad\quad
\quad\quad\quad\quad\;
\varphi_{|\mathcal{F}} = 0 \implies X_\varphi |_\mathcal{F} =0$
\item $\forall F,G \in \mathcal{C}^\infty(\mathcal{M}), \quad\varphi,\psi \in \mathcal{C}^\infty(\mathcal{M}) \quad\quad\quad \varphi_{|\mathcal{F}}=\psi_{|\mathcal{F}}=0 \implies \{F+\varphi, G+\psi\}_{|\mathcal{F}} = \{F,G\}_{|\mathcal{F}}$
\end{enumerate}
Conséquence: on peut définir un crochet de Poisson sur les feuilles, car si on connait $F_{|\mathcal{F}}$ et $G_{|\mathcal{F}}$ on connait $\{F,G\}_{|\mathcal{F}}$
$$\rightsquigarrow \{\cdot,\cdot\}_{\mathcal{F}} : \mathcal{C}^\infty(\mathcal{F})\times\mathcal{C}^\infty(\mathcal{F}) \to \mathcal{C}^\infty(\mathcal{F})
\quad\quad\quad\quad
\mathrm{non}\;\mathrm{degenere}$$
i.e. $\exists \omega_\mathcal{F} \in \Omega^2(\mathcal{F})$; \quad $\{F,G\}_\mathcal{F}=\omega_\mathcal{F}(X_F,X_G)$ et $\d \omega_F = 0$\\ \\
Bref: les feuilles des distributions non-dégénérées dans les variétés de poisson sont des variétés symplectiques. Résultat assez sympa.
\section{Théories de Jauge}
\subsection{Présentation des théories de référence}
\noindent\underline{Exemple}: Maxwell sur $\mathbb{M}_4=:\mathbb{M}$
\begin{align*}
F &= \frac{1}{2}F_{\mu\nu}\d x^\mu \wedge \d x^\nu
& \mathrm{i}.\e. \; F_{\mu\nu}&= \partial_\mu A_\nu - \partial_\nu A_\mu\\
&= \d A &&\\
A &\in \Omega^1(\mathbb{M})&
|F|^2 &= \frac{1}{2}F_{\mu\nu}F^{\mu\nu}\\
\mathcal{A}[A] &= \int_\mathbb{M} \frac{-1}{2}|F|^2 \d^4x&
&=\frac{1}{2}F_{\mu\nu}F_{\mu'\nu'} \eta^{\mu\mu'}\eta^{\nu\nu'}
\end{align*}
Symétrie de Jauge: $A\mapsto A+\d f \quad\quad\quad \mathcal{A}[A+\d f] = \mathcal{A}[A]$\\
Espace des configurations: $\Omega^1(\mathbb{M}) / \d\Omega^0(\mathbb{M})$\\
Noether II $\implies$ E.L. dégénéré.\\

\noindent\underline{Autre exemple}: Maxwell-Dirac
$$\mathcal{A}_\mathrm{Maxwell} + \int_\mathbb{M} \bar \Psi \cancel{\mathcal{D}}\Psi + c \cdot \!\!\!\!\!\!\!\!\!\!\!\!\!\!\underset{\quad\quad\mathrm{terme}\;\mathrm{cubique}}{\bar \Psi \cancel{A}\Psi}$$
Donne une équation d'Euler non-linéaire.\\

\noindent\underline{Yang-Mills pure}:
$A\in\Omega^1(\mathbb{M})\otimes\mathfrak{g}$, pour $\mathfrak{g}$ une algèbre de Lie, le plus souvent parmi:
$$\begin{matrix}
\mathfrak{u}(1) & \leftrightsquigarrow & \mathrm{E}.\mathrm{M}.\\
\mathfrak{su}(2) & \leftrightsquigarrow & \mathrm{weak}\\
\mathfrak{su}(3) & \leftrightsquigarrow & \mathrm{strong}
\end{matrix}$$
Pour choisir un exemple à filer le long de cette section, on peut considérer $\mathfrak{su}(2)$ vu comme:
$$\mathfrak{su}(2) = \mathrm{Vec}\Bigg(
\left(\begin{matrix}
0 & -1 \\
1 & 0
\end{matrix}\right),
\left(\begin{matrix}
0 & i \\
i & 0
\end{matrix}\right),
\left(\begin{matrix}
1 & 0 \\
0 & -1
\end{matrix}\right)\Bigg)
 = \mathrm{Vec}(E_i)$$
 \begin{align*}
 A = A_\mu &\d x^\mu = A^i_\mu E_i \d x^\mu\\
 \mapsto F &= \d A + A\wedge A\\
 &= \d A + \frac{1}{2}[A\wedge A]
 \intertext{appelée ``forme de courbure" (// avec Mauer-Cartan)}
 &= \frac{1}{2}\left(\partial_\mu A_\nu-\partial_\nu A_\mu + [A_\mu,A_\nu]\right)\d x^\mu \wedge \d x^\nu\\
 &= \frac{1}{2} F^i_{\mu\nu} E_ i \d x^\mu \wedge \d x^\nu
 \end{align*}
$$|F|^2 = \frac{1}{2} F^{i\mu\nu}F^j_{\mu\nu} \!\!\!\!\!\!\!\!\!\!\!\!\!\!\!\!\!\!\!\!\!\!\!\!\!\!\!\!\!\!\underset{\quad\quad\quad\quad\quad\mathrm{produit}\;\mathrm{scalaire}\;\mathrm{sur}\;\mathfrak{g}}{K_{ij}}\quad\quad\quad\quad
\mathcal{A}[A]=\int_\mathbb{M} \frac{-1}{2}|F|^2 \d^4 x$$

Pour $g\in\mathcal{C}^\infty(\mathbb{M},G)$, (i.e. $g=\e^\varphi$) on prend la transformation de jauge $A\mapsto g^{-1}Ag+g^{-1}\d g$, et on remarque, évidement, qu'on retrouve $A\mapsto A+\d\varphi$ dans le cas abélien. Si $(K_{ij})$ est invariant par l'action adjointe de $G$ sur $\mathfrak{g}$, alors $\mathcal{A}[g^{-1}Ag+g^{-1}\d g] = \mathcal{A}[A]$. C'est une symétrie de Jauge (en général, non-abélienne).
$$\mathrm{E}.\mathrm{L}.\;: \quad \quad \quad \boxed{\partial_\mu F^{i\mu\nu} - C^i_{jk} A^j_\mu F^{k\mu\nu} =0}$$

On reconnais, dans le premier terme, Maxwell; et dans le second, des termes (interactions) non-linéaires.

\subsection{Géométrie des théories de Jauge: connexion sur un fibré principal}
On peut voir $A$ comme une connexion sur $\mathcal{F}$, un fibré principal au dessus de $X$, groupe de structure de $G$:
$$\begin{matrix}
\mathcal{F}\\
P \downarrow \quad \\
X
\end{matrix}\quad \mathcal{F}=X\times G
\quad \mathrm{Action}\;\mathrm{de}\;G\;\mathrm{sur}\;\mathcal{F}\; \mathrm{a} \;\mathrm{droite}$$
$$\begin{matrix}
\mathcal{F}\times G & \to & \mathcal{F}\\
(z,g) & \mapsto & z \cdot g
\end{matrix}
\quad\quad \rightsquigarrow \quad\quad
\begin{matrix}
\mathcal{F}\times \mathfrak{g} & \to & T\mathcal{F}\\
(z,\xi) & \mapsto & (z,\,z \cdot \xi)
\end{matrix}
\quad\quad \rightsquigarrow \quad\quad
\mathcal{F}_x = P^{-1}(\{x\}) = "\mathrm{Orbite}\;\mathrm{de}\;\mathrm{l}'\mathrm{action}\;\mathrm{de}\; G.
$$
\color{red} METTRE LE DESSIN \color{black}
On appel cette construction une \emph{connexion d'Ehresmann} (connexion sur des fibrés lisses), et est définie rigoureusement par:
$$\forall z \in \mathcal{F} \quad V_z = \mathrm{ker} \; \d P_z \quad \quad \d P_z T_z \mathcal{F} \to T_{P(z)} X$$
Utilisant l'extension naturelle sur les algèbres de Lie, on obtient:
$$z\cdot \xi = \frac{\d}{\d t}\left(z\e^{t\xi}\right)_{|t=0} \in V_z$$

D'où
$V_z = \mathrm{ker} \d P_z = z \cdot \mathfrak{g}$. 
La connexion d'Ehresmann peut être vue comme une distribution $(H_z)_{z\in\mathcal{F}}$ où $H_z \subset T_z \mathcal{F}$ et $H_z \oplus V_z = T_z \mathcal{F}$.

\color{red}METTRE LE DESSIN\color{black}
$$\d P_z |_{H_Z} : H_z \to T_{P(z)}X \; \mathrm{iso}$$
Comment représenter $H_z$? Nous allons construire $\Theta_z: T_z \mathcal{F} \to \mathfrak{g}$ linéaire tel que $\mathrm{ker}\, \Theta_z = H_z$. Notons que $\Theta_z$ est à priori non-unique.\\
\underline{On normalise:} $\Theta_z (z\cdot \xi) = \xi$. Ce qui revient, en gros à dire que la restriction de $\Theta$ à une fibre est (en gros, modulo identification) Mauer-Cartan.\\
\underline{On suppose:} $(H_z)$ invariante par l'action de $G$ i.e. $\iff R_g^* \Theta = \mathrm{Ad}_g^{-1} \Theta = g^{-1} \Theta g$. On pale de forme \emph{equivariante}.\\
\underline{On utilise une connexion "usuelle":} voir exposé d'Ehresmann de 1952 à Bourbaki pour plus d'info.\\
\underline{Trivialisation:} i.e. existence d'une section $\sigma: X \to \mathcal{F}$ (En réalité, il n'en existe pas forcement, mais localement, si, donc on peut voir une trivialisation comme un choix qui pave tout $X$, peu-importe ce qui marche...)
$$\begin{matrix}
X\times G & \to & \mathcal{F}&\\
(x,g) & \mapsto & \sigma(x)\cdot g&\\
(x,g) & \leftarrow & z&\\
&\to & "\mathrm{coordonnees}& (x,g)\in X\times G
\end{matrix}
\quad \quad \color{red}METTRE\;DESSIN
$$
$$\Theta^{-1} = g^{-1} \underset{=A_\mu(x)\d x^\mu}{A(x)} g + g^{-1} \d g$$

Le premier terme est indépendant du degré (c.f. hypothèse d'équivariance) tandis que le second gère la normalisation. Attention: on dirait une symétrie de Jauge, mais il s'agit en fait d'une expression sur les coordonnées. Ici, $g$ est une coordonnée sur $\mathcal{F}$, i.e. une variété telle que $\mathrm{dim}\,\mathcal{F} =\mathrm{dim}\,\mathbb{M}+\mathrm{dim}\,\mathfrak{g}$ et non une application $X \to G$.

Si on change $\sigma \mapsto \tilde \sigma = \sigma \cdot \gamma$ où $\gamma: X \to G$; la transformée de jauge de $A$, $A \mapsto \tilde A$, alors $A\in \Omega^1(X)\otimes \mathfrak{g}$ décrit la connexion d'Ehresmann.\\ \\
\underline{Note de rigueur:} $A\approx P^* a$ pour passer de la version sur $\mathbb{M}$ à $\mathcal{F}$... Mais bon...
$$\d \theta + \frac{1}{2}[\theta\wedge\theta] = g^{-1} \left(\d A +\frac{1}{2}[A\wedge A]\right)g$$
\underline{Note:} non trivial, dans cette égalité se cache l'utilisation de Mauer-Cartan pour annuler les composantes verticales.

\section{Intégrale des Chemins (point de vue de Feynman)}
$$\boxed{\boxed{\int_{\mathrm{Champs}\;\varphi}\!\!\!\!\!\!\!\!\!\!\!\!\!\!\!\D\varphi \quad\e^{\frac{iS(\varphi)}{\hbar}}}}$$
Exemple: $\{\varphi:\mathbb{M}\to\mathbb{C}\}$
$$S(\varphi) = \int_\mathbb{M}\frac{1}{2}|\partial_0 \varphi|^2 - \sum_{a=1}^3 |\partial_a \varphi|^2 - m^2 |\varphi|^2
\quad \quad \underset{\mathrm{E}.\mathrm{L}.}\rightsquigarrow \quad \quad
\Box \varphi + m^2 \varphi = 0 \quad \mathrm{i}.\mathrm{e}. \; \mathrm{Klein}-\mathrm{Gordon}$$
\underline{Problème:} \emph{ça veut dire quoi?}
\subsection{Difficultés et Méthode}
\noindent \underline{Difficultés:}
\begin{enumerate}
\item Le ``$\,i\;$" dans $\e^{iS/\hbar}$ rends déjà les choses compliquées. $\int_\mathbb{R}\d x e^{ix^2}$ est une intégrale oscillante (Fresnel) donc ça converge, mais déjà $\int_{\mathbb{R}^n} \d^nx \e^{i|x|^2}$ est beaucoup plus compliqué et nécessite en général de déformer des contours dans le plan complexe (Rotations de Wick) $\int_{\mathbb{R}^n}\d^nx\e{-\alpha|x|^2}$ avec $\mathrm{Re}(\alpha)>0$ puis faire tendre $\alpha$ vers $i$... Le tout guidé par la seule formule que l'on ait: formule des Gaussiennes.
$$\begin{matrix}
Q & : & \mathbb{R}^n\to \mathbb{R}\\
Q(x) & = & A_{ij}x^ix^j\geq 0
\end{matrix}
\quad \quad \quad
\int_{\mathbb{R}^n}\e^{\frac{1}{2}Q(x)} \d^n x = \frac{(2\pi)^{^n\!/_{\!2}}}{\sqrt{\mathrm{det}(A)}}
$$
\item La dimension infinie des espaces fonctionnels est un gros problème.
$$\mathbb{R}^n \quad \quad \rightsquigarrow \quad \quad \mathcal{E}:= \mathcal{C}^0([0,1],\mathbb{R}^n) \ni \varphi$$
Alors, le ``$\D\varphi$" dans $\int_\mathcal{E}\D\varphi \e^{-Q(\varphi)}$ n'existe pas si on veut une mesure de Lebesgue. On peut résoudre ce problème avec des \emph{mesures de Wienen} mais c'est très subtil de bien choisir $\mathcal{E}$, notamment sa topologie. Et en général, il faut en faire un espace de distributions.
\item Avec un terme d'interaction $\mathcal{I}=\int_\mathcal{E}\D\varphi\e^{-Q(\varphi)/2+I(\varphi)}$, où $I$ est un polynôme de degrés $\geq 3$, c'est la catastrophe, en général plus rien ne converge. On travaille donc uniquement sur des cas particuliers, en petite dimension (de $\mathbb{M}$) ou bien \emph{par perturbation}.\\
Travailler en perturbation, c'est renoncer au calcul de $\mathcal{I}$ et en faire un développement asymptotique en $\varepsilon$ avec:
$$\mathcal{I}_\varepsilon = \int_\mathcal{E}\D\varphi\e^{-Q(\varphi)/2+\varepsilon I(\varphi)}$$
mais du coup, il faut renormaliser...
\item \underline{Idée de la méthode perturbative} illustrée en dimension finie.
$$\langle P \rangle = \frac{
\int_{\mathbb{R}^n} \e^{iA(x,x)/2}P(x)\d^n x
}{
\int_{\mathbb{R}^n} \e^{iA(x,x)/2}\d^n x
} \quad \quad \quad P \in \mathbb{R}[x_i]$$
$$\langle x^1x^2\rangle = \left.\dr{}{J_1}\dr{}{J_2} \e^{A^{-1}(J,J)/2}\right|_{J=0} \quad \quad
\begin{matrix}
A(x,x) & = & A_{ij}x^ix^j\\
A^{-1}(J,J) & = & (A^{-1})^{ij} J_i J_j
\end{matrix}
 \quad \quad J=(J_i) \;\mathrm{coord}\;\mathrm{sur}\;\mathbb{R}^n$$
 \underline{Preuve:}
 \begin{align*}
 W(J):=& \int_{\mathbb{R}^n} \e^{-^1\!/\!_2A(x,x)+\langle J,x \rangle}\d^n x\\
 \dr W {J_i} =& \int_{\mathbb{R}^n} \e^{-^1\!/\!_2A(x,x)+\langle J,x \rangle} x^i\d^n x
 \\
 \dr {^2W} {J_i\partial J_j} =& \int_{\mathbb{R}^n} \e^{-^1\!/\!_2A(x,x)+\langle J,x \rangle} x^ix^j\d^n x\\
 \dr {^2W} {J_i\partial J_j}(0) =& \int_{\mathbb{R}^n} \e^{-^1\!/\!_2A(x,x)} x^ix^j\d^n x\\
 \mathrm{Donc}\; \langle x^ix^j\rangle =& \dr {^2W} {J_i\partial J_j}(0)\\
 \mathrm{Or}\; W(J) =& [...\mathrm{calcul}\;\mathrm{peu}\;\mathrm{passionnant}...]\\ =& \e^{^1\!/\!_2(A^{-1})^{ij}J_iJ_j} \times W(0)\\
 \mathrm{Donc}, \; \langle x^ix^j\rangle =& \left.\dr {^2} {J_i\partial J_j} \e^{^1\!/\!_2(A^{-1})^{ij}J_iJ_j}\right|_{J=0\quad \quad \quad \Box}
 \end{align*}
 Ce calcul se généralise trivialement à $\langle P(x)\rangle = \left.P\left(\dr{}J\right) \e^{^1\!/\!_2A^{-1}(J,J)}\right|_{J=0}$, ce qui permet les développements asymptotiques.\\
 En dimension finie, pour les cas ``gentils" (K-G ou Dirac) on peut faire à peut prêt pareil. Développer devient alors ce qu'on appel la renormalisation.
 \item Problème supplémentaire pour les théories de Jauge: l'analogue de $A$ n'est plus inversible (c'est re-la galère).\\
 \underline{Analogie via Yang-Mills:}
 $$F_{\mu\nu}=(\partial_\mu A_\nu - \partial_\nu A_\mu + ...)$$
 Où les termes ci-dessus sont les termes linéaires, et les termes ``..." sont les non-linéaires.
 $$\mathcal{A}[A] = \int_\mathbb{M} \frac{-1}{4}F_{\mu\nu}F^{\mu\nu} = \int_\mathbb{M}\frac{-1}{4}|\d A|^2 + I(A)$$
 Ici, $|\d A|^2$ présente un caractère dégénéré. Pourquoi? Passons en Fourrier:
 $$A= \mathrm{cste}\times \int (\eta^{\mu\nu} ||p||^2-p^\mu p^\nu) \hat A_\mu \hat A_\nu \quad \mathrm{avec}\quad \hat A_\mu(p) = \int A_\mu(x) \e^{i p_\nu x^\nu /\hbar}\d^n x$$
 Y a un p'tit souci car l'intégrale s'annule sur le cône, mais passons... Le véritable problème est que $A^{-1}$ a pour noyau $\left(\frac{1}{\eta^{\mu\nu} ||p||^2-p^\mu p^\nu}\right)$ et donc n'existe pas!
 $$M^{\mu\nu}(p):=\eta^{\mu\nu}||p^2|| - p^\mu p^\nu \quad \implies \quad M^{\mu\nu}(p) p_\nu = 0$$
 et cela est fondamentalement lié à $A\mapsto A+\d f$...
\end{enumerate}
\subsection{Une construction: l'intégrale de Berezin}
On souhaite construire:
$$\int_{\ppi V} : \left\{\begin{matrix}
\mathcal{C}^\infty(\ppi V) & \to & \mathbb{R}\\
f & \mapsto & \int_{\ppi V} f(\theta) \D \theta^1 ... \D\theta^n
\end{matrix}\right.$$
vérifiant:
\begin{itemize}
\item \underline{Linéarité:} $\int_{\ppi V}$ est linéaire.
\item \underline{Stokes:} $\int_{\ppi V} (\D\theta)^n \dr f {\theta^i} = 0$
\begin{align*}
\implies \int_{\ppi V} (\D \theta)^n f(\theta) &= \int_{\ppi V} (\D\theta)^n f_{1...n} \theta^1...\theta^n\\
&= C f_{1...n}, \quad C\int \mathbb{R}
\end{align*}
\item \underline{Normalisation:} $C=1$
\end{itemize}

\noindent \underline{Petite Bizarrerie:} Formule du changement de variable:
$$\theta = A \tilde \theta, \quad A \in GL(V^*)$$
$$\int_{\ppi V} (\D\theta)^nf(\theta) = \int_{\ppi V} \left(\D\tilde \theta\right)^n (\mathrm{det}\;A)^{-1} f\left(A\cdot\tilde\theta\right)$$
Alors que cette expression devrait avoir un $\mathrm{det}\;A$ à la puissance $+1$ en géométrie classique.\\
$\int_{\ppi V} \D \theta^n...\D\theta^1 f(\theta)$ correspond mathématiquement à $e_n \lrcorner(...(e_1\lrcorner(e_1\lrcorner \alpha))...) = (e_1 \wedge ... \wedge e_n)\lrcorner \alpha.$\\

\noindent \underline{Motivation (supersymétrie):} Superparticule dans une variété Riemannienne $\mathcal{N}$: 
\\
Soit $x:\mathbb{R}\to \mathcal{N}$ un boson et $\psi:\begin{matrix}\mathbb{R}&\to&\ppi T_x\mathcal{N}&\\t&\mapsto&\psi(t)&\in\ppi T_{x(t)}\mathcal{N}\end{matrix}$ un fermion
$$\iff (x,\psi): \mathbb{R} \to \ppi T\mathcal{N}=\left\{(a,v), a\in \mathcal{N}, v\in \ppi T_x\mathcal{N}\right\}$$
où $\ppi T\mathcal{N}$ est un fibré sur $\mathcal{N}$.\\
On note que $\mathcal{C}^\infty(\ppi T\mathcal{N})=\Omega^0(\mathcal{N})$...\\
Supersymétrie: $Q_\eta :
\left\{
\begin{matrix}
x    & \mapsto & x-\eta\psi\\
\psi & \mapsto & \psi+\eta\dot x
\end{matrix}
\right.
$ \quad avec $\eta \in \mathcal{C}^\infty(\ppi\mathbb{R})$ générateur ($\eta^2=0$).\\

\noindent \underline{Exemple}:
$$\mathcal{A}(\dot x, \psi) = \int_\mathbb{R}\d t \frac{1}{2}\left(|\dot x|^2 + \langle\psi,\nabla_{\dot x} \psi\rangle\right)$$
Action invariante (modulo un terme exact) par la symétrie $Q_\eta$.\\

\noindent Formulation en supertemps: $\mathbb{R}^{1/1}$ (le premier $1$ correspond à $t$, le deuxième à $\theta$), tel que $\mathcal{C}^\infty(\mathbb{R}^{1/1})=\mathcal{C}^\infty(\mathbb{R})\oplus\theta\mathcal{C}^\infty(\mathbb{R})$
$$\phi : \begin{matrix}
\mathbb{R}^{1/1} & \to & \mathcal{N}&\\
(t,\theta) & \mapsto & \phi(t,\theta) & = x(t) + \theta \psi(t)\end{matrix}$$
$$\mathcal{A}(x,\psi) = \int\int_{\mathbb{R}^{1/1}} \d t \D\theta \left\langle \left(\dr{} \theta - \theta \dr{} t\right) \phi, \dr\phi t\right\rangle$$
$$Q_\eta : \phi \mapsto \phi + \eta \left(\dr {} \theta + \theta \dr{} t\right)\phi$$

\subsection{Application à Maxwell (vers le Gauge-fixing)}
\quad On cherche à définir:
$\int_{A\in\Omega^1(\mathbb{M})}\e^{\frac i\hbar S(A)}$ avec $S(A):= \int_\mathbb{M} \d^4x \left(\frac{1}{4}F_{\mu\nu}F^{\mu\nu}-j^\mu A_\mu \right)$. Le terme avec les $F_{\mu\nu}$ est quadratique et peut donc être ramené à une gaussienne. On observe la symétrie de jauge: $S(A+\d \varphi) = S(A)$ lorsque $\varphi$ est à décroissance rapide. Cela implique que l'opérateur $Q$ intervenant dans $S$ n'est pas inversible. $S$ est constante sur chaque $A+\d \Omega^{0}_\mathrm{c}(\mathbb{M})$ l'orbite du groupe de jauge; il n'y a donc pas d'oscillations sur cette orbite, et donc pas de problème de définition de $\int \e^{iS(A)}\hbar$. De plus, l'orbite $\approx \d \Omega^{0}(\mathbb{M})$ est un espace de dimension infinie, \emph{mais} $A$ et $A+\d \varphi$ représentent le même état physique. Idée: fixer la jauge.\\

\noindent \underline{Exemple:} On impose $\partial_\mu A^\mu = 0\quad\Big(\iff \d(*A)=0\Big)$ qu'on appel la jauge de Lorentz. Dès lors, $A\mapsto A+\d \varphi$ implique $\partial_\mu A^\mu \mapsto\partial_\mu A^{\mu} + \Box \varphi$. L'unicité de $A$ dans une orbite de Jauge est garantie si on impose des conditions aux bords.\\

\noindent \underline{Caricature en dimension finie:}
\begin{itemize}
\item $\Omega^1(\mathbb{M}) \quad \longrightarrow \quad \mathcal{M}$ variété de dimension $N$.
\item $\Omega^0_\mathrm{c}(\mathbb{M}) \quad \longrightarrow\quad \mathfrak{g}$ algèbre de Lie de dimension $k$.
\item $\{\mathrm{orbites}\}=\Omega^1(\mathbb{M})/\d \Omega^0_\mathrm{c}(\mathbb{M}) \quad \longrightarrow\quad \underline{\mathcal{M}}$ variété de dimension $n=N-k$. (Notons qu'ici $\Omega^{0}_\mathrm{c} \hookrightarrow \Omega^1(\mathbb{M})$ via la différentielle).
\item L'intégrande $\D A \e^{\frac{i}{\hbar}S(A)}\mathcal{O}(A)$ (où $\mathcal{O}$ est une observable, c'est à dire une fonction invariante de Jauge) est une $N$-forme $\omega\in\Omega^N(\mathcal{M})$.
\end{itemize}
\underline{Idée:}
$$\int_\mathcal{M} \omega \quad \rightsquigarrow \quad \int_\Sigma p^*\omega \times \mathrm{Jacobien}$$
avec:
\begin{itemize}
\item $\underline\omega$ une $n$-forme sur le quotient $\underline{\mathcal{M}}$
\item $p: \mathcal{M}\to \underline{\mathcal{M}}$ projection
\item $\Sigma$ une hypersurface de dimension $n$ transverse aux orbites (sections du fibré $\mathcal{M}\underset p \to \underline{\mathcal M}$)\\
\end{itemize}
\emph{procédons par analyse-synthèse}

\noindent \underline{Analyse:}

Supposons qu'il existe une telle forme $\underline \omega \in \Omega^n(\underline{\mathcal{M}})$. On suppose qu'il existe une application de fixation de jauge $F:\mathcal{M}\to G$ (où $G$ est la fibre de $\mathcal{M}\to\underline{\mathcal{M}}$) telle que:
$$\forall x \in \underline{\mathcal M}, \quad F\Big|_{p^{-1}\{x\}} \overset\sim\to G \quad \quad \wedge \quad \quad \mathrm{rg}(\d F, \ p) = N$$
(Analogue pour Maxwell: $F(A)=\partial_\mu A^\mu$, car $F:\Omega^1(\mathbb M) \to \Omega^0(\mathbb M)$\quad)\\
Soit $\theta\in\Omega^k(G)$ tel que $\int_G\theta=1$.
$$\int_{\mathcal M} \omega  \quad \rightsquigarrow \quad \int_{\underline{\mathcal M}} = \int_{\underline{\mathcal M}} \left(\int_{p^{-1}\{x\}}F^*\theta\right)\underline \omega$$
Soient des champs de vecteurs $\left\{\begin{matrix}
Y_1  ...  \;Y_k & \mathrm{tangents}\\
X_1  ...  \;X_n & \mathrm{horizontaux}
\end{matrix}\right.$ aux fibres sur $\mathcal M$, tels que $\bigg(Y_1(z), ... Y_k(z)\bigg)$ base de $T_zp^{-1}\{x\}$ et $(Y_1, ... Y_k, X_1, ... X_n)$ base de $T_z\mathcal M$.
$$\int_{\underline{\mathcal M}} \underline\omega= 
\int_{\underline{\mathcal M}} \int_{p^{-1}\{x\}} \left(F^*\theta(Y_1, ... Y_k)\d y^1\wedge...\wedge\d y^k\right) p^*\underline\omega(X_1,...X_n)\d x^1\wedge...\wedge\d x^n$$
avec $x=p(z)$ et \!$\left\{\begin{matrix}
\d y^1\wedge...\wedge\d y^k (Y_1, ... Y_k) = 1\\
\d x^1\wedge...\wedge\d x^k (X_1, ... X_n) = 1
\end{matrix}\right.$ Or, comme $p_*Y_\alpha = 0$, on a $Y_\alpha\lrcorner p^*\underline\omega=0$ et donc:
\begin{align*}
\int_{\underline{\mathcal M}} \underline\omega &= 
\int_{\mathcal M} \left(F^*\theta\wedge p^*\underline\omega\right)(Y_1, ...Y_k, X_1,...X_n)\d^k y\wedge\d^n x\\
&= 
\int_{\mathcal M} \left[(Y_1...Y_k)\lrcorner(F^*\theta\wedge p^*\underline\omega)\right](X_1,...X_n)\d^k y\wedge\d^n x\\
&= 
\int_{\mathcal M} F^*\theta\wedge p^*\underline\omega
\end{align*}
\underline{Synthèse:}

On part de $\int_{\mathcal M} F^*\theta\wedge p^*\underline\omega$ où $\omega$ est invariante par l'action du groupe de jauge. $\mathfrak{g}$ algèbre de Lie, et représentation $\rho: \begin{matrix}
\mathfrak g\times\mathcal M & \to & T\mathcal M\\
(\xi,z) & \mapsto & \xi \cdot z
\end{matrix}$\\
Soit $(e_1, ...e_k)$ une base de $\mathfrak g$ et $(e^1, ...e^k)$ sa duale, on définit $Y_\alpha(z) = e_\alpha\cdot z$ pour $\alpha \in [\![1,k]\!]$.\\
Hypothèse de symétrie: $\boxed{L_{Y_\alpha}\omega=0}$.\\
On prend une fixation de jauge $F:\mathcal{M}\to \mathfrak{g}$; $\theta \in \Omega^k(\mathfrak g),\; \theta = \varphi e^1\wedge...\wedge e^k$ pour $\varphi\in\mathcal{C}^\infty_\mathrm{c}(\mathfrak g)$.\\
Remarque d'Antoine: \emph{pas clair en général si $F$ est à valeur dans $G$ ou $\mathfrak g$...} on peut alors définir:
$$\int_{\underline{\mathcal M}} \underline\omega := \int_\mathcal M F^*\theta \wedge (Y\lrcorner \omega)$$
où $Y\lrcorner\omega = p^*\omega$ pour faire le lien avec l'analyse, avec évidement $Y=Y_1\wedge...\wedge Y_k$.

\noindent \underline{Lemme:}\\
Si $\mathfrak g$ est unimodulaire (i.e. $c^\beta_{\alpha\beta}=0$, i.e. l'action adjointe $\mathfrak g\to\mathfrak g$ est sans trace) alors:
$$L_{Y\alpha}\omega = 0 \quad \implies\quad
\left\{
\begin{matrix}
\d(Y\lrcorner\omega)=0\quad \;\;\,&\\
Y\lrcorner \omega \; \mathrm{invariante}&
\!\!\! \mathrm{par}\;\mathrm l'\mathrm{action}\;\mathrm{de}\;\mathfrak g
\end{matrix}
\right.$$
\underline{Note:} Si $\mathfrak g$ est unimodulaire, on peut définir la forme $\underline \omega$, comme ça:
$$\int_{\underline{\mathcal M}} \underline\omega := \int_\mathcal M F^*\theta \wedge (Y\lrcorner \omega) \overset{[...]}= \int_\mathcal M (F^*\theta) (Y) \omega$$
En écrivant $\theta = \varphi e^1\wedge...\wedge e^k$, on a :
$$(F^*\theta)(Y) = (\varphi\circ F) \mathrm{det}\left[\dr{(F^\alpha\circ \rho)}{\xi^\beta}\right]=: (\varphi\circ F) \mathrm{det}[A^\alpha_{\;\beta}] \quad \quad A^\alpha_{\;\beta} = e^\alpha \circ \d F_z \circ \rho(e_\beta)$$
$$\mathfrak{g} \begin{matrix}
\overset \rho \to T_z\mathcal M \overset{\d_zF}\to\\
\underset A \longrightarrow
\end{matrix} \mathfrak g$$
\color{red} refaire ce diagramme\color{black}

\noindent \underline{Conclusion:} posant $\omega = P^* \underline{\omega}$ on a
\begin{align}
\int_\mathcal{M} \omega \quad \overset{\mathrm{gauge}\;\mathrm{fix}}{\rightsquigarrow\rightsquigarrow\rightsquigarrow}
\quad & \int_{\underline{\mathcal{M}}} \underline{\omega}\\
=& \int_\mathcal{M} (F^* \theta) \wedge (Y\lrcorner \omega)\label{GF2}\\
=& \int_\mathcal{M} (\varphi\circ F) \mathrm{det} \Big( \d (F\circ\rho)_z\Big) \omega\label{GF3}\\
=& \int_\mathcal{M} \delta_0 (F) (\mathrm{FP}) \omega \label{GF4}
\end{align}

Dans l'équation (\ref{GF2}) $\theta \in \Omega^k(\mathfrak{g})$; $\theta=\varphi\; e^1\wedge...\wedge e^k$ et $Y= Y_1 \wedge ... \wedge Y_k$ avec $Y_\alpha = \rho(e_\alpha)$. Le terme "$\mathrm{det} \Big(\d (F\circ\rho)_z\Big)$" dans (\ref{GF3}) est le déterminant de Faddeev Popov, noté FP. Enfin, le passage de (\ref{GF3}) à (\ref{GF4}) utilise le fait que $A$ peut être vue via le diagramme commutatif suivant:
$$\mathfrak{g}
\begin{matrix}
\overset{\rho_z}\to T_x\mathcal{F}_x \overset{\d F} \to \\
\underset A \longrightarrow
\end{matrix}
\mathfrak{g}$$
\color{red} refaire le diagramme au propre dès que j'ai le temps...\color{black} et pour finir, on remplace $\varphi \in \mathcal{C}^\infty_\mathrm{c}$ par $\delta_0$ sur $\mathfrak{g}$. Pour rappel, on peut voir $\delta_0$ comme limite (au sens des distrib') mais sinon, aussi via Fourrier formel:
$$\delta_0(F) = \frac{1}{(2\pi\hbar)^k}\int_{\mathfrak{g}^*} \d^4\lambda \e^{\frac{i}{\hbar}\langle\lambda,F\rangle}$$
où $\lambda \in \mathfrak{g}^*$ est un multiplicateur de Lagrange. De même FP $= \mathrm{det} \; A$ où $A=\d (F\circ \rho)$.\\

\underline{Remarque}: On a un petit problème avec FP qui n'est pas local en l'espace-temps (.e. si on change $\mathcal M$ par $\Omega^1(\mathbb M)$, par exemple, $\mathrm{det}(A)_{z\in\Omega^1(\mathbb M)}$ ne se calcul pas à partir de la forme locale de $z$)...
\subsection{Un peu de super-calcul}
Soit:
\begin{itemize}
\item $V$ un espace vectoriel de dimension $k$;
\item $\ppi(V^*\oplus V)$ le foncteur de super-parité;
\item $\mathcal{C}^\infty\big(\ppi(V^*\oplus V)\big) = \Lambda^0(V^*\oplus V)^*$ où $\Lambda^0$ est l'algèbre extérieure;
\item $(e_1, ..., e_k)$ une base de $V$, $(e^1, ..., e^k)$ sa base duale (et on les combinera pour les bases des produits tensoriels);
\item $A\in \mathrm{End}(V)\approx V\otimes V^*$ que l'on décompose en $A=A^i_{\;j}e^j\otimes e_i$
\end{itemize}
\underline{Propriété:}
$$
I:= \int_{\ppi(V^*\oplus V)}\D c^k \D \Bar c_k ... \D c^1 \D \Bar c_1 \quad \e^{\frac i \hbar \langle \Bar c, A c \rangle}
= \left(\frac{i}{\hbar}\right)^k \mathrm{det} \; A
$$
Preuve:
\end{document}