\documentclass[a4paper,11pt]{article}
%\usepackage[utf8]{inputenc}

\usepackage{amsmath, amsfonts, amsthm}
\usepackage{tikz}
\usepackage{braket}
\usepackage{cancel}

\renewcommand{\d}{{\mathrm{d}}}
\newcommand{\e}{{\mathrm{e}}}
\newcommand{\dr}[2]{\frac{\partial {#1}}{\partial{#2}}}

\title{Géometrie Différentielle et Physique Mathématique}
\date{Frédéric Hélein}

\usepackage{geometry}
\geometry{a4paper, left=25mm, right=25mm, top=30mm, bottom=25mm}

\renewcommand{\baselinestretch}{1.2}

\begin{document}
\title %ça marche pas
\underline{Objectif}: Atteindre les théories BRST\footnote{BRST: Carlo Becchi, Alain Rouet, Raymond Stora \& Igor Tyutin} et BV\footnote{Igor Batalin \& Grigori Vilkovisky}, théories physiques dévellopées pour quantifier les théories de jauge, tout particulièrement les Yang-Mills mais aussi d'autres.

\section{Calcul des variations}
\subsection{Rappels de base de physique des particules classiques (en formalisme Lagrangien)}
\quad On considère une particule (classique) dans une variété $\mathcal{M}$ de dimension $m$; $I=]t_0,t_1[$ un intervalle réel ouvert (le temps, $t_0, t_1 \in \mathbb{R}\cup \{\pm \infty\}$); et on appel:
\begin{align}
	``\mathrm{Lagrangien}":
		&&L: &&I\times\mathcal{M} \quad&\to\quad \mathbb{R}\\
		&&&& (t,x,v) \quad&\mapsto\quad L(t,x,v)\notag\\
	``\mathrm{Action}":
		&&\mathcal{A}: &&\mathcal{C}^1(I,\mathcal{M}) \quad &\to \quad	\mathbb{R}\\
		&&&&\gamma \quad&\mapsto\quad\mathcal{A}[\gamma]:=\int_I L\big(t,\gamma(t),\dot \gamma(t)\big)\notag
\end{align}
où $L$ est au moins $\mathcal{C}^1$ en $x$ et $\mathcal{C}^2$ en $v$, et où
\begin{equation*}\begin{split}
\delta\mathcal{A}_\gamma[\delta\gamma]
&=\int_I \dr{L}{x^i}(t,\gamma,\dot \gamma)\delta\gamma^i + \dr{L}{v^i}(t,\gamma, \dot \gamma)\frac{\d \delta \gamma^i}{\d t}\\
&=\int_I \frac{\d}{\d t}\left(\dr{L}{v^i}(t,\gamma,\dot \gamma)\delta\gamma^i \right)+ \left(\dr{L}{x^i}(t,\gamma, \dot \gamma) - \frac{\d}{\d t}\dr{L}{v^i}\right)\delta \gamma^i
\end{split}\end{equation*}

\underline{Principe de Maupertuis} (généralisé): On obtient les trajectoires d'une physique classique régie par $L$ en se restreignant à l'ensemble des chemins $\gamma$ tels que $\forall \delta \gamma \quad \delta\mathcal{A}_\gamma[\delta_\gamma]=0$. i.e. ce sont les chemins qui extremisent localement l'action (hors cas physique, on parlera donc simplement de ``points critiques").\\ \\
D'où on dérive le \underline{principe d'Hamiltion}: 
$\forall \delta\gamma \; \mathrm{t}.\mathrm{q}.\; \delta\gamma(t_0)=\delta\gamma(t_1)=0$
\begin{equation*}
\;_{(\mathrm{Maup})} \; \delta\mathcal{A}_\gamma[\delta\gamma] = 0 \quad \quad \iff \quad \quad \boxed{\frac{\d}{\d t}\left(\dr{L}{v_i}(t,\gamma,\dot\gamma)\right) = \dr{L}{x^i}(t,\gamma,\dot\gamma)}_{\quad(\mathrm{E}.\scalebox{0.75}[1.0]{-}\mathrm{L}.)}
\end{equation*}
où l'équation à droite est appelée ``equations d'\underline{Euler-Lagrange}" (E.-L.) (pour une physique de particules). (Existe aussi en version théorie de champs, cf plus tard).

\subsection{1$^\mathrm{er}$ théorème de Noether, symétries et conservation (cas des particules)}
Première difficulté: qu'est-ce qu'une symétrie? Il s'agit, grossièrement d'une action d'un groupe de Lie. (Enfin, d'une algèbre de Lie plutôt...)\\
Version simple:

METTRE LE DESSIN

$$X = X^0(t,x)\frac{\partial}{\partial t} + X^i(t,x) \frac{\partial}{\partial x^i} \quad \quad \quad T:=X^0$$
On note $\Delta_X\subset \mathbb{R}\times(I\times\mathcal{M})$ maximal sur lequel le flot est défini.
\begin{equation*}
	\Phi_X : \left\{\begin{matrix}
	\Delta_X & \to & I\times\mathcal{M} &\\
	(\epsilon,t,x) & \mapsto & \Phi_X(\epsilon,t,x) &=\e^{\epsilon X}(t,x)
	\end{matrix}\right.
\end{equation*}
i.e. $\frac{\partial \Phi}{\partial \epsilon}(\epsilon, t, x) = X(\Phi_X(\epsilon, t, x))$ et $\Phi_X(0,t,x)=(t,x)$
\\
en coord loc, ça donne: $\e^{\epsilon X}(t,x)=\left(\quad t+\epsilon T(t,x),\quad x^i+\epsilon X^i(t,x)\quad \right) + o(\epsilon)$
\\
Action sur $\mathcal{C}^1(I',\mathcal{M})$ où $I'$ est un intervalle compacte de $I$:
\begin{align*}
\gamma &\mapsto \gamma_\epsilon\\
[t_0,t_1] &\mapsto[t_0(\epsilon),t_1(\epsilon)]=[t_0+\epsilon T(t_0,x_0),t_1+\epsilon T(t_1,x_1)]\quad \mathrm{modulo}\;\epsilon
\end{align*}
\begin{align*}
\forall i \in [\![1,n]\!] \quad \gamma^i_{\epsilon}(\Phi_X^0(\epsilon,t,x) &= \Phi_X^i(\epsilon,t,\gamma(t))\\
\gamma_\epsilon &= \gamma + \epsilon\delta\gamma + o(\epsilon)
\end{align*}
\begin{align*}
(\gamma^i+\epsilon\delta\gamma^i)(t+\epsilon T(t,\gamma)) = \gamma^i + \epsilon X^i(t,\gamma) + o(\epsilon)
&\iff \frac{\d \gamma^i}{\d t}T + \delta\gamma^i = X^i\\
&\iff \boxed{\delta \gamma^i = X^i(t,\gamma) - T(t,\gamma)\dot\gamma^i}
\end{align*}
$$X \; \mathrm{symetrie}\;\mathrm{de}\;L \overset{(\mathrm{def})}{\iff} \forall [t_0,t_1]\subset I \quad \int_{t_0(\epsilon)}^{t_1(\epsilon)} L(t,\gamma_\epsilon, \dot\gamma_\epsilon)\d t = \int_{t_0}^{t_1} L(t,\gamma,\dot\gamma)\d t + (\epsilon)$$
\underline{Théorème 1}: Si $X$ est une symétrie et si $\gamma$ est un point critique alors
$$Q_X(t) := \dr{L}{v^i}(t,\gamma,\dot\gamma)X^i(t,\gamma) - \left(\dr{L}{v^i}(t,\gamma,\dot\gamma)\dot\gamma^i - L(t,\gamma,\dot\gamma)\right)T(t,\gamma)$$
est conservé (i.e. $\frac{\d Q}{\d t}=0$).\\
\\
\underline{Remarque}: $Q_X = \dr{L}{v^i} + LT$\\ \\
Preuve du théorème:
$\forall \gamma \in \mathcal{C}^1(I,\mathcal{M})$\\
\begin{align*}
1) \;\mathrm{hypothese}\;\mathrm{de}\;\mathrm{symetrie}\;
&\iff \int_{t_0}^{t_1} L(t,\gamma_\epsilon,\dot\gamma_\epsilon) = \int_{t_0}^{t_1}L(t,\gamma,\dot\gamma) + \epsilon [LT]_{t_0}^{t_1} + \int_{t_0}^{t_1} \left(\dr{L}{x^i}\delta \gamma^i + \dr{L}{v^i}\delta\dot\gamma^i\right) +o(\epsilon)\\
&\iff \int_{t_0}^{t_1} \delta_X L(t,\gamma,\dot\gamma) \d t:=\int_{t_0}^{t_1} \left(\dr{L}{x^i}\delta \gamma^i + \dr{L}{v_i}\delta\dot \gamma^i + \frac{\d (LT)}{\d t}\right)\d t =0
\end{align*}
où $\delta_X L : I\times T\mathcal{M} \to \mathbb{R}$. 
Bref, ``symétrie $\implies \delta_X L = 0$".\\
\\
\underline{Exo}: construire $\delta_X L$ et montrer que ça marche...\\
\\
2) Montrons que $Q$ constant si (et seulement si) $\gamma$ est un point critique.
$$\frac{\delta L}{\delta \gamma^i} := \dr{L}{x^i}(t,\gamma,\dot\gamma) - \frac{\d}{\d t}(\dr{L}{v^i}(t,\gamma,\dot\gamma))$$
D'où EL $\iff \frac{\delta L}{\delta \gamma^i}=0$
\begin{align*}
\dr{L}{x^i}&= \frac{\delta L}{\delta \gamma^i} + \frac{\d}{\d t}(\dr{L}{v^i})\\
\frac{\d Q_X}{\d t}
&= \frac{\d}{\d t} (\dr{L}{v^i}\delta \gamma^i + LT)\\
&= \frac{\d}{\d t} (\dr{L}{v^i})\delta \gamma^i + \dr{L}{v^i} \delta\dot\gamma^i+\frac{\d}{\d t}(LT)\\
&= \dr{L}{x^i} \delta^i+\dr{L}{v^i}\delta \dot \gamma + \frac{\d (LT)}{\d t} - \!\!\!\!\!\!\!\!\!\!\!\!\!\!
\underset{\quad\quad\quad=0\;\mathrm{par}\;(\mathrm{E}.\scalebox{0.75}[1.0]{-}\mathrm{L}.)}{\cancel{\frac{\delta L}{\delta \gamma^i}\delta \gamma^i}}\\
&=0 \quad \mathrm{par} \; \mathrm{symetrie}
\end{align*}
\\
\underline{Variante}: Si $\exists f:I\times\mathcal{M}\to \mathbb{R}$ t.q. $\delta_X L(t,\gamma,\dot\gamma) = \frac{\partial f}{\partial t}(t,x) + v^i \frac{\partial f}{\partial x^i}(t,x)$ ``symétrie modulo un terme exacte" (def) alors la quantité conservée est $(Q_X -f)$.\\

Même si on va aller plus loin dans les théorèmes de Noether plus tard, une bonne référence (historique) est \emph{Les Théorèmes de Noether: Invariance et lois de conservation au XXe siècle} par  Yvette Kosmann-Schwarzbach, éditions de l'école Polytechnique, ISBN: 978-2730211383.


\subsection{Formalisme Hamiltonien}
L'idée est de faire un changement de variable de $T\mathcal{M}$ vers $T^*\mathcal{M}$...

Commençons par def un var symplectique

Un var symplectique $\mathcal{M}$ est une varr munie d'une 2forme $\omega$, $\omega \in \Omega^2(\mathcal{M})$ t.q. 

POINT $\omega$ non dégénérée i.e. $\forall \xi\in T\mathcal{M}, \xi \lrcorner \omega = 0 \implies \xi =0$; \quad \quad $\xi \lrcorner \omega:= \omega(\xi,\cdot)$ (in other books, denoted $\iota_\xi \omega$)

POINT $\d \omega = 0$

Dans des coords loc, $\omega = \sum_{1\leq a_1 < a_2\leq n} \omega_{a_1a_2} \d x^{a_1} \wedge \d x^{a_2}$, et les hypothèses reviennent à dire que rang$(w_{a_1a_2})$ est max, d'où dim$\mathcal{M}$ paire.

Théorème de Darboux tout point admet une carte (et un jeu de coordonnées $(p_i)\!\!\smile\!\!(q^i)$ dessus) dans laquelle $\omega = \d p_i\wedge\d q^i$

Constructions ultra classiques de var symplectiques:

a) $\mathbb{R}^{2n}=\mathbb{R}^n\times\mathbb{R}^n$, on peut donc définir $\omega$ comme dans le théorème de Darboux sur tout $\mathbb{R}^{2n}$.

b) Soit $\mathcal{M}$ une variété de dim $n$, $\exists (T^*\mathcal{M})\to^\pi \mathcal{M} \quad (q,p) \mapsto^\pi q$

Soit $\theta \in \Omega^1 (T^*\mathcal{M})$,
$$\forall \xi \in T_{(p,q)}(T^*\mathcal{M}), \theta_{(q,p)}(\xi) = < p, \d \pi_{(q,p)} \xi >$$
AJOUTER LES FLECHES

en coord loc, $q^i$ sur $\mathbb{M}$ et $p^i$ sur $T_q^*\mathcal{M}$ t.q. $p = p_iq^i$ \, $\theta = p_i \d (q^i\circ \pi)$ où $p_i$ et $q^i$ sont des coord loc sur $T^*\mathcal{M}$

UTILISER LES BONS SYMBOLS

$$\approx p_i\d q^i\in \Omega^1(T^*\mathcal{M})$$
abus de notation, attention, c'est tricky, ça rentre en conflit avec la formule juste avant.

$\omega = \d \theta$ forme symplectique.

Lien entre Lagrangien et géo symplec (eq de Hamilton)
FAIRE LE DIAGRAMME

$(\d q^i)$ Base de $T_q^*\mathcal{M}$, $p\in T^*_q\mathcal{M} \implies p=p_i\d q^i$

d'où $(p_i, q^i)$ coord sur $T^*\mathcal{M}$; enfin, en fait c'est $q^i\circ \pi$ mais bon, c'est l'abus de notation de tout à l'heure.

$$\theta = p_i\d q^i$$

Tranformation de Legendre
$$\forall (t,q) \in \mathbb{R}\times\mathcal{M} \d (L|_{\{t\}\times T_q\mathcal{M}}) =: \dr{L}{v}(t,q,v)$$

en coord loc, $v=v^i\frac{\partial}{\partial q^i} \in T_q\mathcal{M}$
$$\dr{L}{v} = \dr{L}{v^i} \d v^i$$

Hypothèse de Legendre: 
\begin{align*}
&&\mathbb{L}: \mathbb{R}\times T\mathcal{M} \to & \mathbb{R}\times T^*\mathcal{M}\\
&&(t,q,v)\mapsto & (t,q, \dr{L}{v}(t,q,v)) &&
\mathrm{est}\; \mathrm{un}\; \mathrm{diffeo}
\end{align*}

Exemple: $L = \frac{m |v|^2}{2} - V(q)$

Def Hamiltionen
\begin{align*}
H : \mathbb{R}\times T^*\mathcal{M} &\to \mathbb{R}\\
(H\circ\mathbb{L})(t,q,v) &= \dr{L}{v^i}(t,q,v) v^i - L(t,q,v)\\
\iff (\mathrm{implicit})\quad  \mathbb{L}^{-1}&: (t,q,p)\to (t,q,v(t,q,p))
\end{align*}
$$\dr{L}{v^i}(t,q,v(t,q,p)) =: p_i $$
$$H(t,q,p) = p_i v^i(t,q,p) - L(t,q,v(t,q,p))$$

\subsection{retours sur Noether}
$T\frac{\partial}{\partial t} + X^i \frac{\partial}{\partial x^i}$ sur $\mathbb{R}\times\mathcal{M}$ est une symétrie de $L$.

$\implies Q = \dr{L}{v^i}(t,\gamma, \dot\gamma)X^i(t,\gamma) - (\dr{L}{v^i}\dot\gamma^i - L(t,\gamma,\dot\gamma))$ est conservé si $\gamma$ est solution.

$$Q = (p_i \circ \mathbb{L})X^i - (H\circ\mathbb{L})T$$
METTRE LES SOUSTITRES

\begin{align*}
\d H &= v^i \d p_i + \cancel{p_i\d v^i} - \dr{L}{t}(t,q,v) \d t - \dr{L}{q_i}\d q^i - \cancel{\dr{L}{v^i}(t,q,v(t,q,p) \d(v^i)}\\
&= v^i\d p_i - (\dr{L}{t}\circ \mathbb{L}^{-1})\d t - (\dr{L}{q^i}\circ \mathbb{L}^{-1}) \d q^i
\end{align*}
D'où 
\begin{align}
	\dr{H}{t} &= -\dr{L}{t}\circ \mathbb{L}\\
	\dr{H}{q^i} &= - \dr{L}{x^i} \circ \mathbb{L}\\
	\dr{H}{p_i} &= v^i
\end{align}

d'où $\forall \gamma:\mathbb{R} \to \mathcal{M}$
$$\pi = \dr{L}{v}(t,\gamma,\frac{\d \gamma}{\d t})$$

Lemme:
\begin{equation}
\frac{\d}{\d t} (\dr L{v^i} (t,\gamma, \dot \gamma)) = \dr L{q_i} (t,\gamma,\dot\gamma)\quad (\mathrm{EL}) \quad \iff \quad
\begin{split} \frac{\d \gamma^i}{\d t}&=\dr{H}{p_i}(t,\gamma,\pi) \quad (\mathrm{Hq})\\
\frac{\d \pi_i}{\d t}&=-\dr{H}{q^i}(t,\gamma,\pi) \quad (\mathrm{Hp})\end{split}
\end{equation}
Preuve:
$$Hq \iff (t\gamma,\pi) = \mathbb{L}(t,\gamma,\dot\gamma)$$
$$\dr{H}{p_i}(t,\gamma,\pi)=v^i(t,\gamma,\pi)=\frac{\d \gamma^i}{\d t}$$
par def de $v$

Alors, $\pi_i = \dr{L}{v_i}(t,\gamma,\dot\gamma)$
$$\frac{\d \pi}{\d t} = \frac{\d}{d t}(\dr{L}{v_i}) = (EL) \dr{L}{x^i} = -\dr{H}{q^i}$$
Notation:
\begin{align*}
\frac{\d q^i}{\d t} &= \dr H{p_i}\\
\frac{\d p_i}{\d t} &= -\dr H{q^i}
\end{align*}

\subsection{Formulation Géométrique}
$t\mapsto (\gamma(t), \pi(t)) \in \mathcal{C}^1(\mathbb{R},T^*\mathcal{M})$ est solution de Hamiltion
$$\iff \frac{\d}{\d t}(\gamma^i, \pi_i) = (\dr H{p_i},-\dr H{q^i})(\gamma,\pi)$$
ADD COMMENTS

$$X_H = \dr H{p_i} \dr{}{q^i}-\dr H{q^i}\dr{}{p_i}$$
$$\omega = \d p_i \wedge \d q^i$$
\begin{align*}
X_H \lrcorner \omega
&= (\dr H{p_i} \dr{}{q^i}-\dr H{q^i}\dr{}{p_i}) \lrcorner \d p_j\d q^j\\
&= \dr H {p_i} (-\delta^{ij} \d p_j) - \dr H {q^i}(\delta_{ij}\d q^j)\\
&= - (\dr H {p_i} \d p_i + \dr H {q^i} \d q^i)\\
&=\dr H t - \d H
\end{align*}
bref:
$$\boxed{X_H \lrcorner \omega + \d H = \dr H t \d t}$$
Artifice: $T^*(\mathbb{R}\times\mathcal{M}) \supset (\mathbb{R}\times\{0\})\times T^*\mathcal{M}\approx\mathbb{R}\times T^*\mathcal{M} $

On pose alors $q^0 = t$ et sur $T^*(\mathbb{R}\times\mathcal{M})$ on étends $\Tilde \omega:= \d p_0 \wedge \d t + \d p_i \wedge \d q^i$ d'où
$$X_{\Tilde H} \lrcorner \Tilde \omega + \d \Tilde H=0$$
et donc on s'intéresse uniquement à l'hyper-surface $p^0 = H$.

\subsection{Théorème de Noether généraux}
Lagrangien d'ordre quelquonque $r$, i.e. $L(x, \dot x, \ddot x, \dot{\ddot x}\;...\; x^{(r)})$. On travaille sur des champs $u:U\to\mathcal{M}$ où $U=\mathbb{R}$ dans le cas particules.

Jets:

Si $\mathcal{M}$ est varr de dim $k$ et $U$ est un ouvert de $\mathbb{R}^n$, 
$$\mathfrak{j}^r u (x) := (x, u(x), \partial u(x), \partial^2 u(x), \,...\, \partial^r u (x)$$
où $\partial^i := \dr{}{^{\mu_1}...\partial^{\mu_r}} =: \partial_{\mu_1...\mu_r}$
Cas général pour des variétés quelconques:

\begin{align*}
	\mathfrak{j}^0(U,\mathcal{M}) &= U\times\mathcal{M}\\
	\mathfrak{j}^1(U,\mathcal{M}) &= \{(x,y,E), (x,y)\in U\times\mathcal{M}, E \;\mathrm{sev}\;\mathrm{de}\; T_{(x,y)}(U\times\mathcal{M})| \mathrm{dim}E=\mathrm{dim}U \; \d (\pi_{U\times\mathcal{M}\to U})_{(x,y)}: T_{(x,y)}(U\times\mathcal{M})\to T_x\mathcal{M} \; \d (\pi_{U\times\mathcal{M}\to U})_{x,y)}|_E : E\to T_x\mathcal{U}\}\\
	\mathfrak{j}^r(U,\mathcal{M}) &= \mathfrak{j}^1(U, j^{r-1}(U,\mathcal{M}))
\end{align*}

Coord locale sur les jets
$$v^i_{\mu_1...\mu_j} \quad\quad \mathrm{t}.\mathrm{q}. \quad\quad v^i_{\mu_1...\mu_j}(j^ru(x))=\dr{u^i}{x^{\mu_1}...\partial x^{\mu_j}}$$

Lagrangien général d'ordre $r$ sur ``$U\to\mathcal{M}$":
$$L : j^r(U,\mathcal{M}) \to \mathbb{R}$$
$$\mathcal{L}[u] = \int_{U} L(j^r u(x))\d^n x$$
Symétrie infinitésimale $u\mapsto u+\epsilon\delta u + o(\epsilon)$ infinitesimales, générés par un champ de veteurs $Z$ sur $U\times\mathcal{M}$.

Ou plutôt, pour être précis, un champ $Z:j^r(u)\to T(U\times\mathcal{M})$.
$$Z = X^\mu \partial_\mu + Y^i  \partial_i$$
$$\delta u^i = Y^i - \dr{u^i}{x^\mu}X^\mu$$

Thm1: Si $L$ est invariant par $X^\mu\partial_\mu + Y^i\partial_i$ et si $u$ est un point critique de $\mathcal{L}$ alors il lui correspond $J^\mu\partial_\mu$ définit sur $U$ tel que $\dr {J^\mu}{x^\mu}=0$

Ex: $u: \mathbb{R}^n \to \mathbb{R} \quad \Omega\subset \mathbb{R}^n$
$\mathcal{L}[u]=\int_{\Omega} \frac{|\nabla u|^2}2 \d x$ (action de Dirichlet)

$\mathcal{L}[u+\epsilon\varphi] = \int_\omega \frac{|\nabla u|^2}2 + \epsilon <\nabla u, \nabla \varphi> + \epsilon^2 \frac{|\nabla \varphi|^2}2$
 ($\varphi$ supposé à support compacte.)
 
\begin{align*}
\delta \mathcal{L}_u[\varphi] &= \int_\Omega <\nabla u, \nabla \varphi> \d x\\
&= \int_\Omega (\mathrm{div}(\varphi\nabla u) - \varphi \Delta u) \d x\\
&=- \int \varphi \Delta u
\end{align*}

Symétrie par translation $u \mapsto u\circ \tau_\epsilon =: u_\epsilon$; $\tau_\epsilon (x) := x-\epsilon v$.

$u_\epsilon(x) = u(x-\epsilon v) \approx u(x) - \epsilon v^i \dr u {x^i}(x) + o(\epsilon)$

$\delta u = - v^i \dr u {x^i}$

Noether: Si $\delta u = 0$, 
$$\dr L {v_\mu} (x,u,\d u) \dr u {x^\nu} - (L(x,u, \d x) \delta_\mu^\nu) v^\mu = J^\nu$$
alors $\dr{J^\nu}{x^\nu} = 0$

(le prof est pas giga sur de la formule pour $J$, voir la démo qui suit)

Cas particulier:
$r=1$, i.e. $L(x,u,\partial u)$, $X^\mu (x,u), Y^i(x,u)$

$$J^\mu = \dr L {v^i} (x,u,\partial u) Y^i - (\dr L {v^i_\mu} \dr{u^i}{x^\nu} - L \delta_\nu^\mu) X^\nu$$
et EL $\implies \dr{J^\mu}{x^\mu}=0$

\end{document}